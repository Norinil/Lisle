%% -*- LaTeX -*-

\begin{classpage}{Event}

An \class{Event} (short for {\em event variable}) is a
synchronization device that allows threads to suspend execution and
relinquish the processor until the event gets signaled.
The basic operations on events are: signal, notify or broadcast the event,
and wait for the event, suspending the thread's execution until another thread signals the event.

\mansection{Synopsis}
\begin{mansynopsis}
#include <lisle/Event>

class Event
{
public:
  ~Event ()
    throw (permission);
  Event ()
    throw (resource);
  void signal ()
    throw ();
  void notify ()
    throw ();
  void broadcast ()
    throw ();
  void wait ()
    throw (@textnit$thrcancel?);
  void wait (const Duration& duration)
    throw (timeout, @textnit$thrcancel?);
private:
  // @textnit$Disable cloning?
  Event (const Event&);
  Event& operator = (const Event&);
};
\end{mansynopsis}

\mansection{Description}
\begin{mandescription}
  \destructor
  Destroys \farg{this} event variable, freeing the resources it
  might hold. No threads must be waiting on \farg{this} event
  variable on entrance.
  \begin{exception}
    \item[permission] is thrown if there are waiting
      threads for \farg{this} event variable.
  \end{exception}

  \constructor{}
  Initializes \farg{this} event variable.
  \begin{exception}
    \item[resource] is thrown if there are not
      enough system resources to create \farg{this} new event.
  \end{exception}

  \function{void}{signal,notify}{}
  Restarts one of the threads that are waiting on \farg{this}
  event. If no threads are waiting on \farg{this} event,
  nothing happens. If several threads are waiting on \farg{this}
  event, exactly one is restarted, but it is not specified
  which.

  \function{void}{broadcast}{}
  Restarts all threads that are waiting on \farg{this} event
  variable. If no threads are waiting on \farg{this} event,
  nothing happens.

  \function{void}{wait}{}
  Waits for \farg{this}
  event variable to be signaled. The thread's execution is
  suspended and does not consume any CPU time until \farg{this}
  event variable is signaled.\\

  \function{void}{wait}{const Duration\& \farg{duration}}
  Waits on \farg{this} event, as the previous (simple) \code{wait()}
  function does, but this function also bounds the duration of the
  wait. If \farg{this} event has not been signaled before the time
  limit given by \farg{duration},
  exception \code{timeout} is thrown. The \farg{duration}
  argument specifies a relative time and can only be strictly
  positive. In case \farg{duration} is negative or zero, the
  function immediately throws a \code{timeout} exception.
  \begin{exception}
    \item[timeout] is thrown if the wait timed out.
  \end{exception}
\end{mandescription}

\mansection{Cancellation}
Both \code{wait()} member functions are cancellation
points. If a thread is canceled while suspended in one of these
functions, the thread immediately resumes execution and throws a
\lisle internal cancel exception. Thus the complete call stack is
unwound and all object destructors executed.

\mansection{Async-Signal Safety}
The \class{Event} member functions are not async-signal safe, and
should not be called from a signal handler. In particular, calling
either \code{signal()} or \code{broadcast()}
from a signal handler may deadlock the calling thread.

\end{classpage}
