%% -*- LaTeX -*-

\begin{classpage}{Thread}

The \class{Thread} class is a thread startup descriptor. 
Objects of this class are used as first argument to the \code{create} function. 
The class is a container which stores:

\begin{itemize}
  \item The address of the thread's ``main'' function. 
    The function that will run concurrently with the calling thread.
  \item The handle to arguments to pass to the thread on startup, if any.
  \item The thread's startup mode. 
    This mode describes whether the thread starts in suspended mode or in running mode. 
    The default startup mode is the running mode. 
    Table~\ref{tab:Thread::Mode} describes the thread startup modes and their effect on a freshly created thread.
  \item The thread's starting state.
    This state describes whether the thread starts in joinable state or in detached state.
    The default starting state is the joinable state.
    Table~\ref{tab:Thread::State} describes the thread startup states.
\end{itemize}

\begin{table}[htbp]
  \centering
  \begin{tabular}{|l|p{0.7\linewidth}|}
    \hline
    \textbf{Mode} & \textbf{Description} \\
    \hline
    \code{Running} &
    The thread calls its main function as soon as it is created. This
    is the default thread start mode.
    \\
    \hline
    \code{Suspended} &
    The thread suspends itself as soon as it is created, before
    calling its main function. A call to \code{Thrid::resume()}
    is needed to switch the thread back into running state.
    \\
    \hline
  \end{tabular}
  \caption{Thread modes.}
  \label{tab:Thread::Mode}
\end{table}

\begin{table}[htbp]
  \centering
  \begin{tabular}{|l|p{0.7\linewidth}|}
    \hline
    \textbf{State} & \textbf{Description} \\
    \hline
    \code{Joinable} &
    The thread starts in joinable state.
    This is the default thread start state.
    \\
    \hline
    \code{Detached} &
    The thread starts in detached state.
    \\
    \hline
  \end{tabular}
  \caption{Thread states.}
  \label{tab:Thread::State}
\end{table}

\mansection{Synopsis}
\begin{mansynopsis}
#include <lisle/Thread>

class Thread
{
public:
  enum State { Joinable, Detached };
  enum Mode { Running, Suspended };
  typedef void(* Callee)();
  typedef Anondle(* Caller)(void(*)(),Anondle&);
  ~Thread ();
  Thread ();
  Thread (const Thread& that);
  Thread& operator = (const Thread& that);
  Thread (void(* main)(),
          Mode mode = Running, State state = joinable);
  template <typename TAT>
  Thread (void(* main)(Handle<TAT>), const Handle<TAT>& args,
          Mode mode = Running, State state = Joinable);
  template <typename TRT>
  Thread (Handle<TRT>(* main)(),
          Mode mode = Running, State state = Joinable);
  template <typename TRT, typename TAT>
  Thread (Handle<TRT>(* main)(Handle<TAT>), const Handle<TAT>& args,
          Mode mode = Running, State state = Joinable);
  Caller call () const;  
  Callee main () const;
  Anondle& args () const;
  State state () const;
  Mode mode () const;
};
\end{mansynopsis}

\mansection{Description}
\begin{mandescription}
  \destructor
  Destroys \farg{this} thread startup descriptor. Releases all used
  resources.

  \constructor{}
  Constructs \farg{this} thread startup descriptor, setting the
  ``main'' function address to \code{NULL}, and the startup mode to
  \code{Running}.

  \constructor{const Thread\& \farg{that}}
  Constructs \farg{this} thread startup descriptor from \farg{that}
  thread descriptor. All values stored in that
  thread descriptor are cloned into \farg{this} thread
  startup descriptor.

  \operator{Thread\&}{=}{=}{const Thread\& \farg{that}}
  Assigns \farg{this} thread startup descriptor from \farg{that} thread
  descriptor. All values stored in \farg{that}
  thread descriptor are cloned into \farg{this} thread
  startup descriptor. A reference to \farg{this} thread startup
  descriptor is returned for assignment chaining.

\end{mandescription}

\end{classpage}
