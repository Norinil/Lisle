%% -*- LaTeX -*-

\begin{classpage}{Condition}

A \class{Condition} (short for {\em condition variable}) is a
synchronization device that allows threads to suspend execution and
relinquish the processor until some predicate on shared data is
satisfied. The basic operations on conditions are: signal the
condition (when the predicate becomes true), and wait for the
condition, suspending the thread's execution until another thread
signals the condition.

A condition variable must always be associated with a mutex, to avoid
the race condition where a thread prepares to wait on a condition
variable and another thread signals the condition just before the
first thread actually waits on it.

\mansection{Synopsis}
\begin{mansynopsis}
#include <lisle/Condition>

class Condition
{
public:
  ~Condition ()
    throw (permission);
  Condition (Mutex& mutex)
    throw (resource);
  void signal ()
    throw ();
  void notify ()
    throw ();
  void broadcast ()
    throw ();
  void wait ()
    throw (permission, @textnit$thrcancel?);
  void wait (const Duration& duration)
    throw (permission, timeout, @textnit$thrcancel?);
private:
  // @textnit$Disable cloning?
  Condition (const Condition&);
  Condition& operator = (const Condition&);
};
\end{mansynopsis}

\mansection{Description}
\begin{mandescription}
  \destructor
  Destroys \farg{this} condition variable, freeing the resources it
  might hold. No threads must be waiting on \farg{this} condition
  variable on entrance.
  \begin{exception}
    \item[permission] is thrown if there are waiting
      threads for \farg{this} condition variable.
  \end{exception}

  \constructor{Mutex\& \farg{mutex}}
  Initializes \farg{this} condition variable.
  \begin{exception}
    \item[resource] is thrown if there are not
      enough system resources to create \farg{this} new condition.
  \end{exception}

  \function{void}{signal,notify}{}
  Restarts one of the threads that are waiting on \farg{this}
  condition. If no threads are waiting on \farg{this} condition,
  nothing happens. If several threads are waiting on \farg{this}
  condition, exactly one is restarted, but it is not specified
  which.

  \function{void}{broadcast}{}
  Restarts all threads that are waiting on \farg{this} condition
  variable. If no threads are waiting on \farg{this} condition,
  nothing happens.

  \function{void}{wait}{}
  Atomically unlocks the \farg{mutex} given in the constructor
  and waits for \farg{this}
  condition variable to be signaled. The thread's execution is
  suspended and does not consume any CPU time until \farg{this}
  condition variable is signaled. The given \farg{mutex} must be
  locked by the calling thread on entrance to this function, else an
  \class{permission} exceprion is thrown.\\
  Unlocking the given \farg{mutex} and suspending on \farg{this}
  condition variable is done atomically. Thus, if all threads always
  acquire the given \farg{mutex} before signaling the condition, this
  guarantees that \farg{this} condition cannot be signaled (and thus
  ignored) between the time a thread locks the given \farg{mutex} and
  the time it waits on \farg{this} condition variable.
  \begin{exception}
    \item[permission] is thrown if the associated mutex 
	  was not locked by the calling thread.
    \item[thrcancel] is thrown if the thread was canceled.
  \end{exception}

  \function{void}{wait}{const Duration\& \farg{duration}}
  Atomically unlocks the \farg{mutex} given in the constructor
  and waits on \farg{this}
  condition, as the previous (simple) \code{wait()}
  function does, but this function also bounds the duration of the
  wait. If \farg{this} condition has not been signaled before the time
  limit given by \farg{duration},
  the given \farg{mutex} is re-acquired,
  and exception \code{timeout} is thrown. The \farg{duration}
  argument specifies a relative time and can only be strictly
  positive. In case \farg{duration} is negative or zero, the
  function immediately throws a \code{timeout} exception.\\
  \begin{exception}
    \item[permission] is thrown if the associated mutex 
      was not locked by the calling thread.
    \item[timeout] is thrown if the wait timed out.
    \item[thrcancel] is thrown if the thread was canceled.
  \end{exception}
\end{mandescription}

\mansection{Cancellation}
Both \code{wait()} member functions are cancellation
points. If a thread is canceled while suspended in one of these
functions, the thread immediately resumes execution and throws a
\lisle internal cancel exception. Thus the complete call stack is
unwound and all object destructors executed.

\mansection{Async-Signal Safety}
The \code{Condition} member functions are not async-signal safe, and
should not be called from a signal handler. In particular, calling
either \code{signal()} or \code{broadcast()}
from a signal handler may deadlock the calling thread.

\end{classpage}
