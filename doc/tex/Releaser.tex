%% -*- LaTeX -*-

\begin{classpage}{Releaser}

A releaser is an unlock monitor. An unlock monitor has
a constructor taking a mutex as argument. The mutex is unlocked on
monitor construction and locked on monitor destruction. Thus the mutex
is unlocked while the monitor is in scope and locked when the monitor
gets out of scope.

Unlock monitors are used to avoid complicated mutex unlock and lock
call sequences in functions. Construct an unlock monitor at the begin
of a function and the mutex will be unlocked until the function exits
either normally, with a \code{return}, or because of an exception.

\mansection{Synopsis}
\begin{mansynopsis}
#include <lisle/Releaser>

class Releaser
{
public:
  ~Releaser ()
    throw ();
  Releaser (Mutex& mutex)
    throw (permission);
private:
  // @textnit$Disable cloning?
  Releaser (const Releaser&);
  Releaser& operator = (const Releaser&);
};
\end{mansynopsis}

\mansection{Description}
\begin{mandescription}
  \destructor
  Destroys \farg{this} unlock monitor. Acquires the lock on the mutex
  given in the constructor.

  \constructor{Mutex\& \farg{mutex}}
  Constructs \farg{this} unlock monitor. Releases the lock on the given
  \farg{mutex}.
  \begin{exception}
    \item[permission] is thrown if the calling thread does not own the \farg{mutex}.
  \end{exception}
\end{mandescription}

\end{classpage}
