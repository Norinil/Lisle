%% -*- LaTeX -*-

\begin{classpage}{Monitor::Lock}

A \class{Monitor::Lock} is a lock monitor. A lock monitor has a
constructor taking a mutex as argument. The mutex is locked on monitor
construction and unlocked on monitor destruction. Thus the mutex is
locked while the monitor is in scope and unlocked when the monitor
gets out of scope.

Lock monitors are used to avoid complicated mutex lock and unlock call
sequences in functions. Construct a lock monitor at the begin of a
function and the mutex will be locked until the function exits either
normally, with a \code{return}, or because of an exception.

\mansection{Synopsis}
\begin{mansynopsis}
#include <lisle/Monitor>

class Monitor::Lock
{
public:
  ~Monitor::Lock ()
    throw ();
  Monitor::Lock (Mutex& mutex)
    throw (Exception::Thread::Deadlock);
private:
  // @textnit$Disable cloning?
  Monitor::Lock (const Monitor::Lock& monitor);
  Monitor::Lock& operator = (const Monitor::Lock& monitor);
};
\end{mansynopsis}

\mansection{Description}
\begin{mandescription}
  \destructor
  Destroys \farg{this} lock monitor. Releases the lock on the mutex
  given in the constructor.

  \constructor{Mutex\& \farg{mutex}}
  Constructs \farg{this} lock monitor. Acquires the lock on the given
  \farg{mutex}.
  \begin{exception}
    \item[Exception::Thread::Deadlock] is thrown if the given
      \farg{mutex} is of type \typename{Mutex::Detect} and if a calling
      thread already owns the \farg{mutex}.
  \end{exception}
\end{mandescription}

\end{classpage}

%%%%%%%%%%%%%%%%%%%%%%%%%%%%%%%%%%%%%%%%%%%%%%%%%%%%%%%%%%%%%%%%%%%%%%%%%%%%%%%

\begin{classpage}{Monitor::Unlock}

A \class{Monitor::Unlock} is an unlock monitor. An unlock monitor has
a constructor taking a mutex as argument. The mutex is unlocked on
monitor construction and locked on monitor destruction. Thus the mutex
is unlocked while the monitor is in scope and locked when the monitor
gets out of scope.

Unlock monitors are used to avoid complicated mutex unlock and lock
call sequences in functions. Construct an unlock monitor at the begin
of a function and the mutex will be nlocked until the function exits
either normally, with a \code{return}, or because of an exception.

\mansection{Synopsis}
\begin{mansynopsis}
#include <lisle/Monitor>

class Monitor::Unlock
{
public:
  ~Monitor::Unlock ()
    throw ();
  Monitor::Unlock (Mutex& mutex)
    throw (Exception::Thread::Permission);
private:
  // @textnit$Disable cloning?
  Monitor::Unlock (const Monitor::Unlock& monitor);
  Monitor::Unlock& operator = (const Monitor::Unlock& monitor);
};
\end{mansynopsis}

\mansection{Description}
\begin{mandescription}
  \destructor
  Destroys \farg{this} unlock monitor. Acquires the lock on the mutex
  given in the constructor.

  \constructor{Mutex\& \farg{mutex}}
  Constructs \farg{this} unlock monitor. Releases the lock on the given
  \farg{mutex}.
  \begin{exception}
    \item[Exception::Thread::Permission] is thrown if the given
      \farg{mutex} is of type \typename{Mutex::Detect} or
      \typename{Mutex::Recursive} and the calling thread does not own
      the \farg{mutex}.
  \end{exception}
\end{mandescription}

\end{classpage}
