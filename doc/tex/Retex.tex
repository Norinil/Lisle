%% -*- LaTeX -*-

\begin{classpage}{Retex}
  
A retex is a {\em re}cursive mu{\em tex} device, and is useful for
protecting shared data structures from concurrent modifications, and
implementing critical sections and monitors.

A retex has two possible states: unlocked -- not owned by any thread
--, and locked -- owned by one thread. A retex can never be owned by
two different threads simultaneously.  A thread attempting to lock a
retex that is already locked by another thread is suspended until the
owning thread unlocks the retex first.

The retex maintains the concept of a lock count. When a thread
successfully acquires a retex for the first time, the lock count is set
to one. Every time a thread relocks this retex, the lock count is
incremented by one. Each time the thread unlocks the retex, the lock
count is decremented by one. When the lock count reaches zero, the
retex becomes available for other threads to acquire. If a thread
attempts to unlock a retex that it has not locked or a retex which is
unlocked, a permission exception is thrown.

\mansection{Synopsis}
\begin{mansynopsis}
#include <lisle/Retex>

class Retex
{
public:
  ~Retex ()
    throw (permission);
  Mutex ()
    throw (resource);
  bool trylock ()
    throw ();
  void lock ()
    throw ();
  void unlock ()
    throw (permission);
  bool tryacquire ()
    throw ();
  void acquire ()
    throw ();
  void release ()
    throw (permission);
private:
  // @textnit$Disable cloning?
  Retex (const Retex&);
  Retex& operator = (const Retex&);
};
\end{mansynopsis}

\mansection{Description}
\begin{mandescription}
  \destructor
  Destroys \farg{this} retex, freeing all resources it might
  hold. On entrance \farg{this} retex must be unlocked.
  \begin{exception}
    \item[permission] is thrown if \farg{this} retex is in the locked
      state.
  \end{exception}

  \constructor{}
  Constructs \farg{this} retex.
  \begin{exception}
    \item[resource] is thrown if there are not
      enough system resources to create a new retex.
  \end{exception}

  \function{void}{lock,acquire}{}
  Locks \farg{this} retex.\\
  If \farg{this} retex is currently unlocked, it becomes locked and
  owned by the calling thread, and \code{lock} returns
  immediately.\\
  If \farg{this} retex is already locked by another thread,
  \code{lock} suspends the calling thread until \farg{this} retex is
  unlocked.\\
  If \farg{this} retex is already locked by the calling thread, the
  function returns immediately, recording the number of times the
  calling thread has locked \farg{this} retex. An equal number of
  unlock operations must be performed before \farg{this} retex returns
  to the unlocked state.

  \function{bool}{trylock,tryacquire}{}
  These functions behave identically to \code{lock()}, except
  that they don't block the calling thread if \farg{this} retex is
  already locked by another thread. If the calling thread could 
  (re)lock \farg{this} retex \code{true} is returned, else these
  functions return \code{false}.

  \function{void}{unlock,release}{}
  Unlocks \farg{this} retex. On entrance to this function \farg{this}
  retex is assumed to be locked and owned by the calling
  thread.\\
  If \farg{this} retex is  not owned by the calling thread a
  \code{permission} exception is thrown.
  \begin{exception}
    \item[permission] is thrown if \farg{this}
      retex is not owned by the calling thread.
  \end{exception}
\end{mandescription}

\mansection{Cancellation}
  None of the retex functions is a cancellation point, not even
  \code{lock()}, in spite of the fact that it can suspend a
  thread for arbitrary durations. This way, the status of retexes at
  cancellation points is predictable.

\mansection{Async-Signal Safety}
  The retex functions are not async-signal safe. What this means is
  that they should not be called from a signal handler. In particular,
  calling \code{lock()} or \code{unlock()} from a signal
  handler may deadlock the calling thread.

\end{classpage}
