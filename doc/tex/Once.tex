%% -*- LaTeX -*-

\begin{classpage}{Once}

The \class{Once} is a base class for {\em once controls}.  The
purpose of a once control is to ensure that a piece of code is
executed at most once, whatever number of calls to its \code{run}
function, in whichever thread. Once controls can be used for thread
safe initialization for example.

The first time the \code{run} function of a once control is called,
the dervied \code{main} function is called without arguments and the
once control records that the call has been performed. Each subsequent
call to the \code{run} function of the same once control does nothing.

A once control must be a \code{Once} derived class where the
pure virtual \code{main} function is overloaded. The \code{main}
function is the place where the actions to perform once are listed.
The \code{run} function only checks that \code{main} will be run
exactly once, thread safely. The \code{run} function should not be
overloaded, or only with care to call the base class' \code{run}.

\mansection{Synopsis}
\begin{mansynopsis}
#include <lisle/Once>

class Once
{
public:
  virtual ~Once ();
  Once ();
  void run ();
  virtual void main () = 0;
};
\end{mansynopsis}

\mansection{Description}
\begin{mandescription}
  \destructor
  Destroys \farg{this} once control. Releases all resources used by
  \farg{this} control.

  \constructor{}
  Constructs \farg{this} once control.

  \function{void}{run}{}
  Runs \farg{this} once control. Checks thread safely that it is the
  first call and calls the overloaded \code{main} function. If
  \farg{this} once control's \code{run} function has already been
  called once, the function returns immediately.

  \function{void}{main}{}
  This is the pure virtual function that must be overloaded. It is
  the place where the actions to be performed once for \farg{this}
  control must the implemented.
\end{mandescription}

\end{classpage}
