%% -*- LaTeX -*-

\begin{classpage}{Thrid}

Class \code{Thrid} is a thread identifying structure with thread manipulation
member functions. The thread member functions act on the thread
identified by the thread identifier, not the calling thread. On some
systems thread identifiers are called thread handlers. In \lisle the
terms ``thread identifier'' and ``thread handler'' have the same
meaning and can be exchanged (however, there is no dedicated class).

In order to get a thread id do one of the following:
\begin{itemize}
\item Instantiate a \code{Thrid} object. The default constructor
  will initialize the resulting thread id to identify the calling
  thread.
\item Call the \code{create} function to create a new thread
  of execution. The function returns the thread id of the created
  thread. See section~\manref{sec:Creation} for details about
  thread creation.
\end{itemize}

\mansection{Synopsis}
\begin{mansynopsis}
#include <lisle/Thrid>

class Thrid
{
public:
  ~Thrid ();
  Thrid ();
  Thrid (const Thrid& that);
  Thrid& operator = (const Thrid& that);
  operator uint32_t () const;
  bool operator == (const Thrid& id) const;
  bool operator != (const Thrid& id) const;
  void detach ()
    throw (detach);
  void cancel ()
    throw ();
  void resume ()
    throw ();
  Exit join ();
    throw (join, deadlock, thrcancel);
  template <typename TRT>
  Exit join (Handle<TRT>& hretd);
    throw (join, deadlock, thrcancel);
};
\end{mansynopsis}

\mansection{Description}
\begin{mandescription}
  \destructor
  Destroys \farg{this} thread id, releasing all used resources. The
  thread itself is not affected. Only the thread id is destroyed.

  \constructor{}
  Constructs \farg{this} thread id.\\
  The result identifies the calling thread.

  \constructor{const Thrid\& \farg{that}}
  Constructs \farg{this} from \farg{that} thread id.  The data
  stored in \farg{that} thread id is cloned into \farg{this} thread id.

  \operator{Thrid\&}{=}{=}{const Thread::Id\& \farg{that}}
  Assigns \farg{this} from \farg{that} thread id. The data
  stored in \farg{that} thread id is cloned into \farg{this} thread id.

  \operator[const]{}{uint32\_t}{uint32}{}
  Casts \farg{this} thread id into a 32 bit unsigned integer. This
  cast can be used to print out \farg{this} thread id in numeric
  format. Most debuggers show the running threads by a number. This
  cast returns the same number for the thread identified by
  \farg{this} thread id.

  \operator[const]{bool}{==}{==}{const Thrid\& \farg{that}}
  Compares \farg{this} with \farg{that} rvalue thread id.
  Returns \code{true} if both ids identify the same thread,
  \code{false} if not.

  \operator[const]{bool}{!=}{"!=}{const Thrid\& \farg{that}}
  Compares \farg{this} with \farg{that} rvalue thread id. Returns
  \code{true} if both ids identify different threads, \code{false}
  if not.

  \function{Exit}{join}{Handle<TRT>\& hretd}
  Suspends the execution of the calling thread until \farg{this} thread terminates, either by thread termination or cancellation.
  \\
  The returned \code{Exit} stores the mean by which \farg{this} thread terminated:
  \begin{description}
    \item[\code{terminated}] if \farg{this} thread terminated normally or
      if \farg{this} thread was exited with a call to \code{exit},
    \item[\code{canceled}] if \farg{this} thread was canceled.
    \item[\code{exceptioned}] if an uncaught exception was thrown in \farg{this} thread.
  \end{description}
  The optional \farg{hretd} allows the calling thread to retrieve a handle to data eventually returned by \farg{this} thread when it terminated.
  See section~\manref{sec:Termination} for details about the \code{exit} function and how it allows a thread to return data to a joining thread.
  \\
  The thread with \farg{this} id must be in the joinable state: it must not have been detached calling \code{detach} or created in detached state.
  If \farg{this} thread is not in joinable state a \class{join} exception with reason \code{join::detached} is thrown.
  \\
  When a joinable thread terminates, its memory resources (thread descriptor and stack) are not released until another thread performs a \code{join} on it.
  Therefore, a joinable thread {\bf must} be joined in order to avoid memory leaks.
  \\
  At most one thread can join \farg{this} thread. 
  If another thread is already waiting for termination of \farg{this} thread a \class{join} exception with reason \code{join::duplicate} is thrown.
  \begin{exception}
    \item[join] is thrown if joining \farg{this} thread failed.\\
      The exception's reason of failure can be:
      \begin{exreason}
        \item[join::missing] if \farg{this} thread no longer exists.
        \item[join::detached] if \farg{this} thread is in detached state.
        \item[join::duplicate] if another thread is already joining \farg{this} thread.
      \end{exreason}
    \item[deadlock] is thrown if \farg{this} thread is the calling thread.
      Self joining would deadlock \farg{this} thread.
  \end{exception}

  \function{void}{detach}{}
  Puts \farg{this} thread in detached state.
  This guarantees that the memory resources consumed by \farg{this} thread will be released immediately when \farg{this} thread terminates. 
  However, this prevents other threads from synchronizing on the termination of \farg{this} thread by joining it.
  \\
  A thread can be created initially in the detached state, by setting the \code{Thread::State} to \code{Thread::Detached}.
  In contrast method \code{detach} applies to threads created in the joinable state, and which need to be put in detached state later.
  \\
  After \code{detach} completes, subsequent attempts to join \farg{this} thread will fail with an exception.
  \\
  If another thread is already joining \farg{this} thread at the time \code{detach} is called, the call does nothing and leaves \farg{this} thread in the joinable state.
  \begin{exception}
    \item[detach] is thrown if detaching \farg{this} thread failed.
      \\
      The exception's reason of failure can be:
      \begin{exreason}
        \item[detach::missing] if \farg{this} thread no longer exists.
        \item[detach::detached] if \farg{this} thread is already in the detached state.
      \end{exreason}
  \end{exception}

  \function{void}{cancel}{}
  Depending on its settings, the target thread can then either ignore the request, or defer it till it reaches a cancellation point.
  \\
  When a thread eventually honors a cancellation request, it performs as if \code{exit} has been called at that point. 
  See section~\manref{sec:Cancellation} for more information on thread cancellation.

  \function{void}{resume}{}
  Resumes the execution of \farg{this} suspended thread.
\end{mandescription}

\mansection{Cancellation} 
\code{Thrid::join()} is a cancellation point. 
If a thread is canceled while suspended in a call to \code{Thrid::join()}, the thread execution resumes immediately and the cancellation is executed without waiting for \farg{this} thread to terminate. 
If cancellation occurs during \code{Thrid::join()}, the thread to be joined remains not joined.

\end{classpage}
