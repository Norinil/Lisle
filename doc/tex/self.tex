%% -*- LaTeX -*-

%%=============================================================================

\begin{manpage}{Suspension}
\let\savedmanlayout=\manlayout
\renewcommand{\manlayout}{vcompact}

A thread can relinquish the processor voluntarily.
\\
It can call \code{yield} to relinquish the processor without blocking.
The thread is moved to the end of the queue for its static priority and a new thread gets to run.
\\
It can call \code{sleep} to relinquish the processor for a given duration.
The thread remains unscheduled for the duration of the sleep.

\mansection{Synopsis}
\begin{mansynopsis}
#include <lyric/self>

void yield ();

void sleep (const Duration& duration);
\end{mansynopsis}

\mansection{Description}
\begin{mandescription}
  \function{void}{yield}{}
    The calling thread relinquishes the processor, giving another thread a higher chance to run.
    If the current thread is the only thread in the highest priority list at that time, this thread will continue to run after a call to \code{yield()}.

  \function{void}{sleep}{const Duration\& \farg{duration}}
    This function causes the calling thread to be suspended from execution until the specified \farg{duration} has elapsed.
    The suspension time may be longer than requested due to scheduling of other activity by the system.
    The sleeping thread does not consume processing resources.
\end{mandescription}

\renewcommand{\manlayout}{\savedmanlayout}
\end{manpage}

%%=============================================================================

\begin{manpage}{Cancellation}
\let\savedmanlayout=\manlayout
\renewcommand{\manlayout}{vcompact}

Cancellation is the mechanism by which a thread can terminate the execution of another thread. 
Functions to \code{set} and \code{get} the thread's cancellation state.
A thread's cancellation state tells how the thread will honor cancellation requests.
Table~\ref{tab:Cancellation} lists the possible thread cancellation states and describes how a thread with a given cancellation descriptor will act on cancellation requests.
\\
\lisle implements a synchronous cancellation scheme only.
This means that a thread with enabled cancellation still needs to perform \code{testcancel} calls from time to time to actually honor a cancellation request.
\\
See also section~\manref{sec:Thrid}, \code{Thrid::cancel()}, for a description on sending a cancellation request to a thread.

\begin{table}[htbp]
  \centering
  \begin{tabular}{|l|l|}
    \hline
    \textbf{Value} & \textbf{Description} \\
    \hline
    \code{cancel::enable} &
    The thread will honor cancellation requests.
    \\
    \hline
    \code{cancel::disable} &
    The thread will ignore cancellation requests.
    \\
    \hline
  \end{tabular}
  \caption{Cancellation states.}
  \label{tab:Cancellation}
\end{table}

\mansection{Synopsis}
\begin{mansynopsis}
#include <lisle/self>

void set (const cancel::State& cancellation);
void get (cancel::State& cancellation);
void testcancel ();
\end{mansynopsis}

\mansection{Description}
\begin{mandescription}
  \function{void}{set}{const cancel::State\& \farg{cancellation}}
  Sets the calling thread's \farg{cancellation} state.

  \function{void}{get}{cancel::State\& \farg{cancellation}}
  Gets the calling thread's \farg{cancellation} state.
  
  \function{void}{testcancel}{}
  Tests for a pending cancellation request and eventually executes it by throwing a \code{thrcancel} exception.
  The purpose if this function is to introduce explicit cancellation checks in long sequences of code that do not call a cancellation point function otherwise.
  \\
  Throwing an exception ensures proper stack unwinding and instances destruction.
  It is important to let exception \code{thrcancel} flow through without being caught.
  See section~\manref{sec:Exceptions.Internal} for details on best practice with \lisle internal exceptions.
\end{mandescription}

\renewcommand{\manlayout}{\savedmanlayout}
\end{manpage}

%%=============================================================================

\begin{manpage}{Termination}
\let\savedmanlayout=\manlayout
\renewcommand{\manlayout}{vcompact}

A thread can terminate by calling the \code{exit} function.
An optional return value can be given in a handle by the terminating thread for an eventual joining thread.
A joining thread gets a \code{terminated} value as return of the \code{join} function.
\\
To guarantee proper stack unwinding and instances destruction \lisle implements the explicit thread termination by throwing a \code{threxit} exception.
It is important to let exception \code{threxit} flow through without being caught.
See section~\manref{sec:Exceptions.Internal} for details on best practice with \lisle internal exceptions.

\mansection{Synopsis}
\begin{mansynopsis}
#include <lyric/self>

void exit ();

template <typename TRT>
void exit (const Handle<TRT>& hretd);
\end{mansynopsis}

\mansection{Description}
\begin{mandescription}
  \function{void}{exit}{}
    This function causes the calling thread to terminate.
    A joining thread gets a \code{terminated} code.

  \function{void}{exit}{const Handle<TRT>\& \farg{hretd}}
    This function causes the calling thread to terminate and return some data stored in \farg{hretd} to a joining thread.
    A joining thread gets a \code{terminated} code.
\end{mandescription}

\renewcommand{\manlayout}{\savedmanlayout}
\end{manpage}

%%=============================================================================
