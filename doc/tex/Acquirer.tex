%% -*- LaTeX -*-

\begin{classpage}{Acquirer}

An acquirer is a lock monitor. A lock monitor has a
constructor taking a mutex as argument. The mutex is locked on monitor
construction and unlocked on monitor destruction. Thus the mutex is
locked while the monitor is in scope and unlocked when the monitor
gets out of scope.

Lock monitors are used to avoid complicated mutex lock and unlock call
sequences in functions. Construct a lock monitor at the begin of a
function and the mutex will be locked until the function exits either
normally, with a \code{return}, or because of an exception.

\mansection{Synopsis}
\begin{mansynopsis}
#include <lisle/Acquirer>

class Acquirer
{
public:
  ~Acquirer ()
    throw ();
  Acquirer (Mutex& mutex)
    throw (deadlock);
private:
  // @textnit$Disable cloning?
  Acquirer (const Acquirer&);
  Acquirer& operator = (const Acquirer&);
};
\end{mansynopsis}

\mansection{Description}
\begin{mandescription}
  \destructor
  Destroys \farg{this} lock monitor. Releases the lock on the mutex
  given in the constructor.

  \constructor{Mutex\& \farg{mutex}}
  Constructs \farg{this} lock monitor. Acquires the lock on the given
  \farg{mutex}.
  \begin{exception}
    \item[deadlock] is thrown if the calling thread already owns the \farg{mutex}.
  \end{exception}
\end{mandescription}

\end{classpage}
