%% -*- LaTeX -*-

%% -*- LaTeX -*-

\documentclass[10pt]{book}
\usepackage{manual}
\usepackage{a4}
\usepackage{enumerate}
\usepackage{makeidx}
\ifpdfman
  \usepackage[pdftex]{graphicx}
  \pdfcompresslevel=9
  \usepackage{minitoc}
  \usepackage[pdftex,colorlinks]{hyperref}
\else
  \usepackage[dvips]{graphicx}
  \usepackage{minitoc}
  \usepackage{hyperref}
\fi

%% Graphics stuff
\graphicspath{{../fig/}}
\ifhtml
  \DeclareGraphicsExtensions{.png}
\else\ifpdf
  \DeclareGraphicsExtensions{.pdf}
\else
  \DeclareGraphicsExtensions{.eps}
\fi\fi

%% Mini TOC stuff
\ifpdf
\else
  \renewcommand{\mtcfont}{\rm}
  \renewcommand{\mtcSfont}{\rm}
  \setlength{\mtcindent}{1.5em}
  \nomtcrule
\fi

%% URL and Hyper References stuff
\urlstyle{sf}

%% Local commands not to be defined in manual style
%% -*- LaTeX -*-

\usepackage{xspace}
\newcommand\lyric{\textsc{Lyric}\xspace}
\newcommand\lisle{\textsc{Lisle}\xspace}

\newcommand\nsThreadNote[2]{%
The \code{Thread::#1} #2 is actually a #2 \code{#1}, part of the
\code{Thread} name space. When using the \code{Thread} name space
\code{Thread::} can be omitted as far as there is no naming ambiguity.
}

\newcommand\mfdExceptionWho{%
  \function[const]{void*}{who}{}
    Returns the address of the object which threw \farg{this}
    exception. Usually the this pointer of the object.
}


%% Begin to index
\makeindex

%% Allow large interword spacing
\sloppy


\begin{document}
\dominitoc

%==============================================================================
\begin{mantitlepage}
  \title{%
    \vspace*{13ex}
    {\Huge \bf \lisle}\\
    \vspace*{7ex}
    {\huge C++ Thread Library}\\
    \vspace*{5ex}
    {\Huge Documentation}
  }
  \author{\Large The \lisle development team}
  \date{\Large \today}
\end{mantitlepage}

\newpage
\thispagestyle{empty}
\newsavebox{\mancopybox}
\newlength{\mancopyboxheight}
\newlength{\mancopyboxvspace}
\sbox{\mancopybox}{\parbox{\linewidth}{%
\hrule
\vspace*{3ex}
Copyright \copyright\ 2002--2013, The \lisle development team.
Permission is granted to copy, distribute and/or modify this document
under the terms of the GNU Free Documentation License, Version 1.2 or
any later version published by the Free Software Foundation; with no
Invariant Sections, no Front-Cover Texts, and no Back-Cover Texts.  A
copy of the license is included in the appendix entitled ``Terms and
Conditions'', section ``Document Usage'' (Appendix~\manref{app:FDL}).
\vspace*{3ex}
\hrule
}}
\settototalheight{\mancopyboxheight}{\usebox{\mancopybox}}
\setlength{\mancopyboxvspace}{\textheight}
\addtolength{\mancopyboxvspace}{-\mancopyboxheight}
\vspace*{\mancopyboxvspace}
\noindent
\usebox{\mancopybox}

%% -*- LaTeX -*-

\newpage
\setcounter{page}{1}
\pagenumbering{roman}

\ifpdf
\else
  \dominitoc
\fi
\tableofcontents

\newpage
\setcounter{page}{1}
\pagenumbering{arabic}

%------------------------------------------------------------------------------

%==============================================================================
\chapter{Introduction}
\label{cha:Introduction}

\lisle is a cross platform object-oriented C++ thread library. The
POSIX thread model and definitions were chosen as design template for
\lisle, but with C++ extensions and features use in mind. \lisle
is more than a C++ wraper around native thread APIs. Currently
\lisle has been ported to POSIX and Win32 native threads.

This library features cancellable and suspendable threads, and various
synchronization devices. This includes semaphores, mutexes, condition
variables, barriers, and other more complex components.
Chapter~\ref{cha:Synchronization-Devices} gives the complete list and
APIs for \lisle synchronization devices.

Existing thread APIs do not provide a consistent and portable means
for exercising control over concurrent threads of execution. For
instance, each platform has its own specialized methods for error
handling, thread cancellation and termination. \lisle includes an
elegant method of safely terminating threads without the complications
of using cancellation handlers or other similar
constructs. Chapter~\ref{cha:Thread-Management} decribes the
primitives and APIs for \lisle thread management.

\lisle can detect various concurrency and synchronization faults in
programs and signals them through C++ exceptions. All synchronization
devices are protected against destruction while at least one thread
still owns it. The library can also detect and signal deadlock
situations. \lisle assists developers as much as possible in order
to provide the best as possible final concurrent
software. Chapter~\ref{cha:Exceptions} lists all exceptions that can
be thrown by \lisle.
%------------------------------------------------------------------------------

%==============================================================================
\chapter{Design Choices}
\label{cha:Design-Choices}

\manminitoc
\noindent
As in every real world engineering project, \lisle{} was designed and
implemented with compromises to the original ideal software that we
had in mind. In this chapter we explain and justify the design and
implementation choices we made for \lisle{}. This chapter is not a
justification for a bad design, but to explain what choices were made
according to the limitations found in current state of the art
computer technology, languages, and third party libraries, on top of
which \lisle{} is built.

\section{Cancellation}
\label{sec:Design-Cancellation}

\lisle provides synchronous thread cancellation. You may wonder why
despite POSIX thread provides asynchronous thread cancellation
\lisle doesn't. This is because the choice was made to impose no
cancellation handler management to the developpers, but to use plain
C++ destruction mechanisms to clean up when a thread honors a
cancellation request.

Indeed, C++ comes with object destructors which are automatically
called when an object exits a scope block. It is a very reliable
mechanism one can build on. Every time an object gets out of focus,
either by regular scope exit or when an exception is thrown, the
object's destructor is called. It is in the destructor that you
implement object cleanup and resources release.

When a \lisle thread honors a cancellation request it uses the
regular C++ stack unwinding and automatic destructor calls. In fact,
in \lisle thread cancellation is implemented by means of a cancel
exception throwing. The advantage is that the aformentioned C++ stack
unwinding mechanism is invoked to cleanup, the drawback being some
restrictions on exception specifications and catching.

Functions where a cancellation request can be honored {\bf can not}
have an explicit C++ exception specification. Indeed, the \lisle's
internal exceptions used to cancel and/or exit a thread would never
match your exception specifications since you don't know the names of
\lisle's internal exceptions. Specifing C++ exceptions in the
cancellable path of a thread will lead to an unexpected exception
being thrown, what is not what you expect.

Furthermore in functions where a cancellation request can be honored
no generic catching (with \code{catch(...)}) can be used. Using a
generic exception catching would catch the \lisle thread management
exception as well as all other exceptions, without filtering. Thus you
would break \lisle's thread management chain and never be informed
what happened to your thread. Plus the thread's internal state
descriptor would no longer be coherent with the thread's actual state.

Using the C++ exception mechanism to implement thread cancellation
forbids the handling of asynchronous cancellation requests. Indeed, it
is not possible to throw a C++ exception in a signal handler (Unix) or
very unsafe in an event handler (Win32). Moreover, imagine what would
happen if the thread to cancel is in a function (say \code{malloc}).
There is no way to predict what would happen when a program jumps out
of a C function without leting this function terminate properly.

Let's take \code{malloc} as an example. This function reserves some
memory and returns a pointer to the memory block. It will first place
a system call to get a memory block, then do some error checking, and
finally assign and return a pointer to the memory block. Imagine the
thread calling malloc is cancelled between the memory reservation
system call and the error checking. The pointer to return would never
be assigned, and we would never know if the memory reservation
succeeded or failed. The process would continue to run, but its state
would no longer be 100\% defined, which is not to recommend for a
state machine (as every piece of software is).
%------------------------------------------------------------------------------

%==============================================================================
\chapter{Synchronization Devices}
\label{cha:Synchronization-Devices}

\manminitoc
\noindent
The synchronization devices is a set of C++ classes used to synchronize
threads. Thread synchronization is used to avoid data race conditions,
to provide flow control mechanisms, and much more.

%% -*- LaTeX -*-

\begin{classpage}{Mutex}
  
A mutex is a {\em mut}ual {\em ex}clusion device, and is useful for
protecting shared data structures from concurrent modifications, and
implementing critical sections and monitors.

A mutex has two possible states: unlocked -- not owned by any thread
--, and locked -- owned by one thread. A mutex can never be owned by
two different threads simultaneously.  A thread attempting to lock a
mutex that is already locked by another thread is suspended until the
owning thread unlocks the mutex first.

The mutex has error checking enabled. If a thread attempts to relock
a mutex that it has already locked, a deadlock exception is thrown.
If a thread attempts to unlock a mutex that it has not locked or a mutex which is unlocked, a permission exception is thrown.

\mansection{Synopsis}
\begin{mansynopsis}
#include <lisle/Mutex>

class Mutex
{
public:
  ~Mutex ()
    throw (permission);
  Mutex ()
    throw (resource);
  bool trylock ()
    throw ();
  void lock ()
    throw (deadlock);
  void unlock ()
    throw (permission);
  bool tryacquire ()
    throw ();
  void acquire ()
    throw (deadlock);
  void release ()
    throw (permission);
private:
  // @textnit$Disable cloning?
  Mutex (const Mutex&);
  Mutex& operator = (const Mutex&);
};
\end{mansynopsis}

\mansection{Description}
\begin{mandescription}
  \destructor
  Destroys \farg{this} mutex, freeing all resources it might
  hold. On entrance \farg{this} mutex must be unlocked.
  \begin{exception}
    \item[permission] is thrown if \farg{this} mutex was
      not unlocked before being destroyed.
  \end{exception}

  \constructor{}
  Constructs \farg{this} mutex.
  \begin{exception}
    \item[resource] is thrown if there are not
      enough system resources to create a new mutex.
  \end{exception}

  \function{void}{lock,acquire}{}
  Locks \farg{this} mutex.\\
  If \farg{this} mutex is currently unlocked, it becomes locked and
  owned by the calling thread, and \code{lock} returns
  immediately.\\
  If \farg{this} mutex is already locked by another thread,
  \code{lock} suspends the calling thread until \farg{this} mutex is
  unlocked.\\
  If \farg{this} mutex is already locked by the calling thread, the
  function throws a \code{deadlock} exception and returns immediately.
  \begin{exception}
    \item[deadlock] is thrown if a calling thread tries to lock
    \farg{this} mutex when it is already owned by the thread.    
  \end{exception}

  \function{bool}{trylock,tryacquire}{}
  These functions behave identically to \code{lock()}, except
  that they don't block the calling thread if \farg{this} mutex is
  already locked by a thread (including the calling thread). If the
  calling thread could lock \farg{this} mutex \code{true} is returned,
  else these functions return \code{false}.

  \function{void}{unlock,release}{}
  Unlocks \farg{this} mutex. On entrance to this function \farg{this}
  mutex is assumed to be locked and owned by the calling
  thread.\\
  If \farg{this} mutex is not owned by the calling thread an
  \code{permission} exception is thrown.
  \begin{exception}
    \item[permission] is thrown if the calling thread does not own
    \farg{this} mutex.
  \end{exception}
\end{mandescription}

\mansection{Cancellation}
  None of the mutex functions is a cancellation point, not even
  \code{lock()}, in spite of the fact that it can suspend a
  thread for arbitrary durations. This way, the status of mutexes at
  cancellation points is predictable.

\mansection{Async-Signal Safety}
  The mutex functions are not async-signal safe. What this means is
  that they should not be called from a signal handler. In particular,
  calling \code{lock()} or \code{unlock()} from a signal
  handler may deadlock the calling thread.

\end{classpage}

%% -*- LaTeX -*-

\begin{classpage}{Retex}
  
A retex is a {\em re}cursive mu{\em tex} device, and is useful for
protecting shared data structures from concurrent modifications, and
implementing critical sections and monitors.

A retex has two possible states: unlocked -- not owned by any thread
--, and locked -- owned by one thread. A retex can never be owned by
two different threads simultaneously.  A thread attempting to lock a
retex that is already locked by another thread is suspended until the
owning thread unlocks the retex first.

The retex maintains the concept of a lock count. When a thread
successfully acquires a retex for the first time, the lock count is set
to one. Every time a thread relocks this retex, the lock count is
incremented by one. Each time the thread unlocks the retex, the lock
count is decremented by one. When the lock count reaches zero, the
retex becomes available for other threads to acquire. If a thread
attempts to unlock a retex that it has not locked or a retex which is
unlocked, a permission exception is thrown.

\mansection{Synopsis}
\begin{mansynopsis}
#include <lisle/Retex>

class Retex
{
public:
  ~Retex ()
    throw (permission);
  Mutex ()
    throw (resource);
  bool trylock ()
    throw ();
  void lock ()
    throw ();
  void unlock ()
    throw (permission);
  bool tryacquire ()
    throw ();
  void acquire ()
    throw ();
  void release ()
    throw (permission);
private:
  // @textnit$Disable cloning?
  Retex (const Retex&);
  Retex& operator = (const Retex&);
};
\end{mansynopsis}

\mansection{Description}
\begin{mandescription}
  \destructor
  Destroys \farg{this} retex, freeing all resources it might
  hold. On entrance \farg{this} retex must be unlocked.
  \begin{exception}
    \item[permission] is thrown if \farg{this} retex is in the locked
      state.
  \end{exception}

  \constructor{}
  Constructs \farg{this} retex.
  \begin{exception}
    \item[resource] is thrown if there are not
      enough system resources to create a new retex.
  \end{exception}

  \function{void}{lock,acquire}{}
  Locks \farg{this} retex.\\
  If \farg{this} retex is currently unlocked, it becomes locked and
  owned by the calling thread, and \code{lock} returns
  immediately.\\
  If \farg{this} retex is already locked by another thread,
  \code{lock} suspends the calling thread until \farg{this} retex is
  unlocked.\\
  If \farg{this} retex is already locked by the calling thread, the
  function returns immediately, recording the number of times the
  calling thread has locked \farg{this} retex. An equal number of
  unlock operations must be performed before \farg{this} retex returns
  to the unlocked state.

  \function{bool}{trylock,tryacquire}{}
  These functions behave identically to \code{lock()}, except
  that they don't block the calling thread if \farg{this} retex is
  already locked by another thread. If the calling thread could 
  (re)lock \farg{this} retex \code{true} is returned, else these
  functions return \code{false}.

  \function{void}{unlock,release}{}
  Unlocks \farg{this} retex. On entrance to this function \farg{this}
  retex is assumed to be locked and owned by the calling
  thread.\\
  If \farg{this} retex is  not owned by the calling thread a
  \code{permission} exception is thrown.
  \begin{exception}
    \item[permission] is thrown if \farg{this}
      retex is not owned by the calling thread.
  \end{exception}
\end{mandescription}

\mansection{Cancellation}
  None of the retex functions is a cancellation point, not even
  \code{lock()}, in spite of the fact that it can suspend a
  thread for arbitrary durations. This way, the status of retexes at
  cancellation points is predictable.

\mansection{Async-Signal Safety}
  The retex functions are not async-signal safe. What this means is
  that they should not be called from a signal handler. In particular,
  calling \code{lock()} or \code{unlock()} from a signal
  handler may deadlock the calling thread.

\end{classpage}

%% -*- LaTeX -*-

\begin{classpage}{Acquirer}

An acquirer is a lock monitor. A lock monitor has a
constructor taking a mutex as argument. The mutex is locked on monitor
construction and unlocked on monitor destruction. Thus the mutex is
locked while the monitor is in scope and unlocked when the monitor
gets out of scope.

Lock monitors are used to avoid complicated mutex lock and unlock call
sequences in functions. Construct a lock monitor at the begin of a
function and the mutex will be locked until the function exits either
normally, with a \code{return}, or because of an exception.

\mansection{Synopsis}
\begin{mansynopsis}
#include <lisle/Acquirer>

class Acquirer
{
public:
  ~Acquirer ()
    throw ();
  Acquirer (Mutex& mutex)
    throw (deadlock);
private:
  // @textnit$Disable cloning?
  Acquirer (const Acquirer&);
  Acquirer& operator = (const Acquirer&);
};
\end{mansynopsis}

\mansection{Description}
\begin{mandescription}
  \destructor
  Destroys \farg{this} lock monitor. Releases the lock on the mutex
  given in the constructor.

  \constructor{Mutex\& \farg{mutex}}
  Constructs \farg{this} lock monitor. Acquires the lock on the given
  \farg{mutex}.
  \begin{exception}
    \item[deadlock] is thrown if the calling thread already owns the \farg{mutex}.
  \end{exception}
\end{mandescription}

\end{classpage}

%% -*- LaTeX -*-

\begin{classpage}{Releaser}

A releaser is an unlock monitor. An unlock monitor has
a constructor taking a mutex as argument. The mutex is unlocked on
monitor construction and locked on monitor destruction. Thus the mutex
is unlocked while the monitor is in scope and locked when the monitor
gets out of scope.

Unlock monitors are used to avoid complicated mutex unlock and lock
call sequences in functions. Construct an unlock monitor at the begin
of a function and the mutex will be unlocked until the function exits
either normally, with a \code{return}, or because of an exception.

\mansection{Synopsis}
\begin{mansynopsis}
#include <lisle/Releaser>

class Releaser
{
public:
  ~Releaser ()
    throw ();
  Releaser (Mutex& mutex)
    throw (permission);
private:
  // @textnit$Disable cloning?
  Releaser (const Releaser&);
  Releaser& operator = (const Releaser&);
};
\end{mansynopsis}

\mansection{Description}
\begin{mandescription}
  \destructor
  Destroys \farg{this} unlock monitor. Acquires the lock on the mutex
  given in the constructor.

  \constructor{Mutex\& \farg{mutex}}
  Constructs \farg{this} unlock monitor. Releases the lock on the given
  \farg{mutex}.
  \begin{exception}
    \item[permission] is thrown if the calling thread does not own the \farg{mutex}.
  \end{exception}
\end{mandescription}

\end{classpage}

%% -*- LaTeX -*-

\begin{classpage}{Semaphore}

Semaphores are counters for resources shared between threads. The
basic operations on semaphores are: increment the counter atomically,
and wait until the counter is non-null and decrement it atomically.

\mansection{Synopsis}
\begin{mansynopsis}
#include <lisle/Semaphore>

namespace lisle {
class Semaphore
{
public:
  ~Semaphore ()
    throw (permission);
  Semaphore (size_t value=0)
    throw (resource);
  static size_t max ()
    throw ();
  size_t value () const
    throw ();
  bool trywait ()
    throw ();
  void wait ()
    throw ();
  void post ()
    throw (overflow);
private:
  // @textnit$Disable cloning?
  Semaphore (const Semaphore&);
  Semaphore& operator = (const Semaphore&);
};
}
\end{mansynopsis}

\mansection{Description}
\begin{mandescription}
  \destructor
  Destroys \farg{this} semaphore, freeing all resources it might
  hold. No threads should be waiting on \farg{this} semaphore at the
  time of destruction.
  \begin{exception}
    \item[permission] is thrown if there are blocked
      threads waiting on \farg{this} semaphore.
  \end{exception}

  \constructor{size\_t \farg{value} = 0}
  Constructs \farg{this} semaphore. The count associated with
  \farg{this} semaphore is initially set to the given
  \farg{value}. If no value is given the count is set to zero.
  \begin{exception}
    \item[resource] is thrown if there are not
      enough system resources to create \farg{this} new semaphore.
  \end{exception}

  \function[static]{size\_t}{max}{}
  Returns the maximum count a semaphore can store.

  \function[const]{size\_t}{value}{}
  Returns the current count of \farg{this} semaphore.

  \function{bool}{trywait}{}
  This function is a non-blocking variant of 
  \code{wait()}. If \farg{this} semaphore has a non-zero
  count, the count is atomically decreased and \code{true} is
  returned, else \code{false} is returned.

  \function{void}{wait}{}
  Suspends the calling thread until \farg{this} semaphore has non-zero count.
  The count of \farg{this} semaphore is atomically decreased on wait termination.
  
  \function{void}{post}{}
  Atomically increases the count of \farg{this} semaphore.
  \begin{exception}
    \item[overflow] is thrown if after incrementation,
      \farg{this} semaphore value would exceed the system's maximum
      value for semaphore count. The semaphore count is left
      unchanged in this case. The system's maximum semaphore count
      value is system dependent.
  \end{exception}
\end{mandescription}

\mansection{Cancellation}
\code{Semaphore::wait()} is a cancellation point.

\mansection{Async-Signal Safety}
  The semaphore functions are not async-signal safe. What this means is
  that they should not be called from a signal handler.

\end{classpage}

%% -*- LaTeX -*-

\begin{classpage}{Condition}

A \class{Condition} (short for {\em condition variable}) is a
synchronization device that allows threads to suspend execution and
relinquish the processor until some predicate on shared data is
satisfied. The basic operations on conditions are: signal the
condition (when the predicate becomes true), and wait for the
condition, suspending the thread's execution until another thread
signals the condition.

A condition variable must always be associated with a mutex, to avoid
the race condition where a thread prepares to wait on a condition
variable and another thread signals the condition just before the
first thread actually waits on it.

\mansection{Synopsis}
\begin{mansynopsis}
#include <lisle/Condition>

class Condition
{
public:
  ~Condition ()
    throw (permission);
  Condition (Mutex& mutex)
    throw (resource);
  void signal ()
    throw ();
  void notify ()
    throw ();
  void broadcast ()
    throw ();
  void wait ()
    throw (permission, @textnit$thrcancel?);
  void wait (const Duration& duration)
    throw (permission, timeout, @textnit$thrcancel?);
private:
  // @textnit$Disable cloning?
  Condition (const Condition&);
  Condition& operator = (const Condition&);
};
\end{mansynopsis}

\mansection{Description}
\begin{mandescription}
  \destructor
  Destroys \farg{this} condition variable, freeing the resources it
  might hold. No threads must be waiting on \farg{this} condition
  variable on entrance.
  \begin{exception}
    \item[permission] is thrown if there are waiting
      threads for \farg{this} condition variable.
  \end{exception}

  \constructor{Mutex\& \farg{mutex}}
  Initializes \farg{this} condition variable.
  \begin{exception}
    \item[resource] is thrown if there are not
      enough system resources to create \farg{this} new condition.
  \end{exception}

  \function{void}{signal,notify}{}
  Restarts one of the threads that are waiting on \farg{this}
  condition. If no threads are waiting on \farg{this} condition,
  nothing happens. If several threads are waiting on \farg{this}
  condition, exactly one is restarted, but it is not specified
  which.

  \function{void}{broadcast}{}
  Restarts all threads that are waiting on \farg{this} condition
  variable. If no threads are waiting on \farg{this} condition,
  nothing happens.

  \function{void}{wait}{}
  Atomically unlocks the \farg{mutex} given in the constructor
  and waits for \farg{this}
  condition variable to be signaled. The thread's execution is
  suspended and does not consume any CPU time until \farg{this}
  condition variable is signaled. The given \farg{mutex} must be
  locked by the calling thread on entrance to this function, else an
  \class{permission} exceprion is thrown.\\
  Unlocking the given \farg{mutex} and suspending on \farg{this}
  condition variable is done atomically. Thus, if all threads always
  acquire the given \farg{mutex} before signaling the condition, this
  guarantees that \farg{this} condition cannot be signaled (and thus
  ignored) between the time a thread locks the given \farg{mutex} and
  the time it waits on \farg{this} condition variable.
  \begin{exception}
    \item[permission] is thrown if the associated mutex 
	  was not locked by the calling thread.
    \item[thrcancel] is thrown if the thread was canceled.
  \end{exception}

  \function{void}{wait}{const Duration\& \farg{duration}}
  Atomically unlocks the \farg{mutex} given in the constructor
  and waits on \farg{this}
  condition, as the previous (simple) \code{wait()}
  function does, but this function also bounds the duration of the
  wait. If \farg{this} condition has not been signaled before the time
  limit given by \farg{duration},
  the given \farg{mutex} is re-acquired,
  and exception \code{timeout} is thrown. The \farg{duration}
  argument specifies a relative time and can only be strictly
  positive. In case \farg{duration} is negative or zero, the
  function immediately throws a \code{timeout} exception.\\
  \begin{exception}
    \item[permission] is thrown if the associated mutex 
      was not locked by the calling thread.
    \item[timeout] is thrown if the wait timed out.
    \item[thrcancel] is thrown if the thread was canceled.
  \end{exception}
\end{mandescription}

\mansection{Cancellation}
Both \code{wait()} member functions are cancellation
points. If a thread is canceled while suspended in one of these
functions, the thread immediately resumes execution and throws a
\lisle internal cancel exception. Thus the complete call stack is
unwound and all object destructors executed.

\mansection{Async-Signal Safety}
The \code{Condition} member functions are not async-signal safe, and
should not be called from a signal handler. In particular, calling
either \code{signal()} or \code{broadcast()}
from a signal handler may deadlock the calling thread.

\end{classpage}

%% -*- LaTeX -*-

\begin{classpage}{Event}

An \class{Event} (short for {\em event variable}) is a
synchronization device that allows threads to suspend execution and
relinquish the processor until the event gets signaled.
The basic operations on events are: signal, notify or broadcast the event,
and wait for the event, suspending the thread's execution until another thread signals the event.

\mansection{Synopsis}
\begin{mansynopsis}
#include <lisle/Event>

class Event
{
public:
  ~Event ()
    throw (permission);
  Event ()
    throw (resource);
  void signal ()
    throw ();
  void notify ()
    throw ();
  void broadcast ()
    throw ();
  void wait ()
    throw (@textnit$thrcancel?);
  void wait (const Duration& duration)
    throw (timeout, @textnit$thrcancel?);
private:
  // @textnit$Disable cloning?
  Event (const Event&);
  Event& operator = (const Event&);
};
\end{mansynopsis}

\mansection{Description}
\begin{mandescription}
  \destructor
  Destroys \farg{this} event variable, freeing the resources it
  might hold. No threads must be waiting on \farg{this} event
  variable on entrance.
  \begin{exception}
    \item[permission] is thrown if there are waiting
      threads for \farg{this} event variable.
  \end{exception}

  \constructor{}
  Initializes \farg{this} event variable.
  \begin{exception}
    \item[resource] is thrown if there are not
      enough system resources to create \farg{this} new event.
  \end{exception}

  \function{void}{signal,notify}{}
  Restarts one of the threads that are waiting on \farg{this}
  event. If no threads are waiting on \farg{this} event,
  nothing happens. If several threads are waiting on \farg{this}
  event, exactly one is restarted, but it is not specified
  which.

  \function{void}{broadcast}{}
  Restarts all threads that are waiting on \farg{this} event
  variable. If no threads are waiting on \farg{this} event,
  nothing happens.

  \function{void}{wait}{}
  Waits for \farg{this}
  event variable to be signaled. The thread's execution is
  suspended and does not consume any CPU time until \farg{this}
  event variable is signaled.\\

  \function{void}{wait}{const Duration\& \farg{duration}}
  Waits on \farg{this} event, as the previous (simple) \code{wait()}
  function does, but this function also bounds the duration of the
  wait. If \farg{this} event has not been signaled before the time
  limit given by \farg{duration},
  exception \code{timeout} is thrown. The \farg{duration}
  argument specifies a relative time and can only be strictly
  positive. In case \farg{duration} is negative or zero, the
  function immediately throws a \code{timeout} exception.
  \begin{exception}
    \item[timeout] is thrown if the wait timed out.
  \end{exception}
\end{mandescription}

\mansection{Cancellation}
Both \code{wait()} member functions are cancellation
points. If a thread is canceled while suspended in one of these
functions, the thread immediately resumes execution and throws a
\lisle internal cancel exception. Thus the complete call stack is
unwound and all object destructors executed.

\mansection{Async-Signal Safety}
The \class{Event} member functions are not async-signal safe, and
should not be called from a signal handler. In particular, calling
either \code{signal()} or \code{broadcast()}
from a signal handler may deadlock the calling thread.

\end{classpage}

%% -*- LaTeX -*-

\begin{classpage}{Shorex}

A shorex is a {\em sh}ared {\em or} {\em ex}clusive access
device, and is usefull for protecting shared data structures from
concurrent modifications.
In literature shorex is also called read/write-lock, or rwlock

A shorex has three possible states: unlocked -- not owned by any
thread --, shared locked -- owned by one or more threads --, and
exclusively locked -- owned by exactly one thread. A shorex can be
owned by multiple threads in shared mode, and owned by a single thread
in exclusive mode. A thread attempting to lock an exclusively locked
shorex, or a thread attempting to exclusive lock a locked shorex is
suspended until the shorex returns to the unlocked state.

\mansection{Synopsis}
\begin{mansynopsis}
#include <lisle/Shorex>

namespace lisle {
enum Access { shared, exclusive };
class Shorex
{
public:
  ~Shorex ()
    throw (permission);
  Shorex ()
    throw (resource);
  bool trylock (Access access)
    throw ();
  void lock (Access access)
    throw (deadlock);
  void unlock ()
    throw (permission);
  bool tryacquire (Access access)
    throw ();
  void acquire (Access access)
    throw (deadlock);
  void release ()
    throw (permission);
private:
  // @textnit$Disable cloning?
  Shorex (const Shorex&);
  Shorex& operator = (const Shorex&);
};
}
\end{mansynopsis}

\mansection{Description}
\begin{mandescription}
  \destructor
  Destroys \farg{this} shorex, freeing all resources it might hold. On
  entrance \farg{this} shorex must be unlocked by all owners.
  \begin{exception}
    \item[permission] is thrown if \farg{this}
      shorex was not unlocked by all owners.
  \end{exception}

  \constructor{}
  Constructs \farg{this} shorex.
  \begin{exception}
    \item[resource] is thrown if there are not
      enough system resources to create \farg{this} shorex.
  \end{exception}

  \function{void}{lock,acquire}{Access \farg{access}}
  Applies a lock on \farg{this} shorex.\\
  The given \farg{access} influences the locking as follows:
  \begin{description}
    \item[\code{shared}] applies a shared lock to \farg{this}
      shorex. The calling thread acquires \farg{this} shorex if it is
      not already held exclusively by another thread. Otherwise the
      calling thread blocks until it can acquire the lock.\\
      If the calling thread already holds a lock on \farg{this} shorex
      a deadlock exception is thrown.
    \item[\code{exclusive}] applies an exclusive lock to
      \farg{this} shorex. The calling thread acquires \farg{this}
      shorex if no other thread holds \farg{this} shorex (shared or
      exclusive). Otherwise the calling thread blocks until it can
      acquire the lock.\\
      If the calling thread already holds a lock on \farg{this} shorex
      a deadlock exception is thrown.
  \end{description}
  \begin{exception}
    \item[deadlock] is thrown if a calling thread
      already holding a lock on \farg{this} shorex attempts to relock
      it.
  \end{exception}

  \function{bool}{trylock,tryacquire}{Access \farg{access}}
  These functions behave identically to
  \code{Shorex::lock(\farg{access})}, except that they don't block the
  calling thread if \farg{this} shorex can not be locked . If the
  calling thread could lock \farg{this} shorex \code{true} is
  returned, else these functions return
  \code{false}.\\
  Note that contrary to \code{Shorex::lock()} nothing tells if the
  locking of \farg{this} shorex failed because of locking logic or
  lack of resources.

  \function{void}{unlock,release}{}
  Unlocks \farg{this} shorex. On entrance to this function \farg{this}
  shorex is assumed to be locked by the calling thread.
  \begin{exception}
    \item[permission] is thrown if \farg{this}
      shorex was not locked by the calling thread.
  \end{exception}
\end{mandescription}

\mansection{Cancellation}
\code{Shorex::lock(exclusive)} is a cancellation point.

\mansection{Async-Signal Safety}
The \code{Shorex} member functions are not async-signal safe, and
should not be called from a signal handler. In particular, calling
\code{Shorex::lock()} or \code{Shorex::unlock()} from a signal handler
may deadlock the calling thread.

\end{classpage}

%% -*- LaTeX -*-

\begin{classpage}{Barrier}

Barriers are multiple threads synchronisation devices. A barrier
blocks waiting threads until the given number of clients have reached
it. Barrier objects are reusable. This means that the same barrier
object can be waited for sequentially multiple times by a given
thread.

\mansection{Synopsis}
\begin{mansynopsis}
#include <lisle/Barrier>

class Barrier
{
public:
  ~Barrier ()
    throw (permission);
  Barrier (size_t clients)
    throw (resource);
  void wait ()
    throw ();
private:
  // @textnit$Disable cloning?
  Barrier (const Barrier& barrier);
  Barrier& operator = (const Barrier& barrier);
};
\end{mansynopsis}

\mansection{Description}
\begin{mandescription}
  \destructor
  Destroys \farg{this} barrier, releasing all resources it might
  hold. No threads must be waiting on \farg{this} barrier on
  entrance.
  \begin{exception}
    \item[permission] is thrown if there are
      waiting threads for \farg{this} barrier.
  \end{exception}

  \constructor{size\_t \farg{clients}}
  Constructs \farg{this} barrier with the given number of
  \farg{clients}. The created barrier will block waiting threads
  until the given number of \farg{clients} is reached.
  \begin{exception}
    \item[resource] is thrown if there are not
      enough system resources to create \farg{this} new barrier.
  \end{exception}

  \function{void}{wait}{}
  Suspends the calling thread until the expected number of \farg{client} threads are waiting on \farg{this} barrier. 
  Once the expected number of \farg{client}s threads have reached \farg{this} barrier all waiting threads are restarted.
\end{mandescription}

\mansection{Cancellation}
None of the barrier functions is a cancellation point.

\mansection{Async-Signal Safety}
The \code{Barrier} member functions are not async-signal safe, and
should not be called from a signal handler. Calling
\code{Barrier::wait()} from a signal handler may deadlock the
calling thread.

\end{classpage}

%------------------------------------------------------------------------------

%==============================================================================
\chapter{Thread Mangement}
\label{cha:Thread-Management}

\manminitoc
\noindent
The thread management is a set of C++ classes and functions to handle
threads. Thread management provides tools to start, stop, identify,
etc, threads.

%% -*- LaTeX -*-

%%=============================================================================

\begin{manpage}{Creation}
\let\savedmanlayout=\manlayout
\renewcommand{\manlayout}{vcompact}

Function \code{create} creates a new thread of control that executes concurrently with the calling thread.
The thread's main function and arguments are given by means of a \class{Thread} argument, while the thread scheduling policy is given with an optional \class{Schedule} argument.

A total of four different types of thread functions can be created,
for threads with or without arguments, or for threads returning data
or threads which return nothing:
\begin{enumerate}
  \item \code{void \textit{thread} ()}
  \item \code{void \textit{thread} (Handle<\textit{TAT}>)}
  \item \code{Handle<\textit{TRT}> \textit{thread} ()}
  \item \code{Handle<\textit{TRT}> \textit{thread} (Handle<\textit{TAT}>)}
\end{enumerate}
Arguments and/or returns are passed through handles (see section~\manref{sec:Handle}).
Handles are kind of smart pointers with
thread safe memory allocation and deallocation. If more complex
synchronization than memory management is needed in thread arguments
or returns, data stored in the handle will need a synchronization
device, like one documented in
Chapter~\ref{cha:Synchronization-Devices}.

\mansection{Synopsis}
\begin{mansynopsis}
#include <lisle/create>

Thrid create (const Thread& start,
              const Schedule& schedule = Schedule())
  throw (resource, permission, virthread);
\end{mansynopsis}

\mansection{Description}
\begin{mandescription}
  \function{Thrid}{create}{%
    const Thread\& \farg{start},%
    const Schedule\& \farg{schedule}}
  Creates a new thread of control that executes concurrently with the
  calling thread. Returns a \code{Thrid} which identifies the
  created thread. Don't loose this \code{Thrid} if you intend to
  join the created thread (what you have to if the created thread is
  not in detached state). Argument \farg{start} describes the thread's
  main function and arguments, while the optional argument
  \farg{schedule} describes the scheduling policy
  in which the created thread should run.
  \begin{exception}
    \item[resource] is thrown if there are
      not enough system resources to create the new thread. This
      happens if either too many threads are running in the same
      process, or if there is a shortage in memory resources.
    \item[virthread] is thrown if
      \farg{start}\code{.main()} returns \code{NULL}. In other words
      if a default \farg{start} argument was given, or if a
      \code{NULL} function pointer was given to construct
      \farg{start}.
    \item[permission] is thrown if the
      process owner isn't allowed to run a thread with a given
      scheduling policy given in the \farg{schedule} argument. Only
      the super user can run threads whith a scheduling policy other
      than \code{Schedule::Regular}.
  \end{exception}
\end{mandescription}

\mansection{Example}
The following example shows the four different types of thread
functions that can be created, how to start them, and if needed how to
give them arguments, and if they return a result how to retrieve the
handle to the return.

\VerbatimInput[frame=lines,labelposition=topline,label=\textnit{Thread
  creation},framesep=3ex,xleftmargin=2.5em,numbers=left,tabsize=2]
  {../example/create-thread.cpp}

\renewcommand{\manlayout}{\savedmanlayout}
\end{manpage}

%%=============================================================================

% \newpage
% \index{Thread::exit|(}

% \section{Thread::exit}
% \label{sec:Thread::exit}

% \nsThreadNote{exit}{function}

% \begin{mansec}{Synopsis}
% \begin{Synopsis}
% #include <lyric/Thread>

% void Thread::exit (void* retdataptr = NULL);
% \end{Synopsis}
% \end{mansec}

% \begin{mansec}{Description}
%   \begin{manfunc}{void Thread::exit (void* \farg{retdataptr})}
%   \end{manfunc}
% \end{mansec}

% \index{Thread::exit|)}

% %%=============================================================================

% \newpage
% \index{Thread::set {\it (cancel behaviour)}|(}

% \section{Thread::set {\it (cancel behaviour)}}
% \label{sec:Thread::set(Thread::Cancel)}

% \code{Thread::set(\farg{cancel})} sets the calling thread's cancel
% behaviour. See~\lhmref{sec:Thread::Cancel} for the
% \code{Thread::Cancel} documentation.
% behaviour.

% \nsThreadNote{set({\it cancel behaviour})}{function}

% \begin{mansec}{Synopsis}
% \begin{Synopsis}
% #include <lyric/Thread>

% void Thread::set (const Thread::Cancel& cancel);
% \end{Synopsis}
% \end{mansec}

% \begin{mansec}{Description}
%   \begin{manfunc}{void Thread::set (const Thread::Cancel\& \farg{cancel})}
%     Set the calling thread's cancel behaviour to the given
%     \farg{cancel} argument.
%   \end{manfunc}
% \end{mansec}

% \index{Thread::set {\it (cancel behaviour)}|)}

% %%=============================================================================

% \vspace*{10ex}
% \index{Thread::get {\it (cancel behaviour)}|(}

% \section{Thread::get {\it (cancel behaviour)}}
% \label{sec:Thread::get(Thread::Cancel)}

% \code{Thread::get(\farg{cancel})} grabs the calling thread's cancel
% behaviour. See~\lhmref{sec:Thread::Cancel} for the
% \code{Thread::Cancel} documentation.

% \nsThreadNote{get({\it cancel behaviour})}{function}

% \begin{mansec}{Synopsis}
% \begin{Synopsis}
% #include <lyric/Thread>

% void Thread::get (Thread::Cancel& cancel);
% \end{Synopsis}
% \end{mansec}

% \begin{mansec}{Description}
%   \begin{manfunc}{void Thread::get (Thread::Cancel\& \farg{cancel})}
%     Grabs the calling thread's cancel behaviour and returns it in
%     \farg{cancel}.
%   \end{manfunc}
% \end{mansec}

% \index{Thread::get {\it (cancel behaviour)}|)}

% %%=============================================================================

% \newpage
% \index{Thread::testcancel|(}

% \section{Thread::testcancel}
% \label{sec:Thread::testcancel}

% \code{Thread::cancel} does nothing except testing for pending
% cancellation and executing it. Its purpose is to introduce explicit
% checks for cancellation in long sequences of code that do not call
% cancellation point functions otherwise.

% \nsThreadNote{testcancel}{function}

% \begin{mansec}{Synopsis}
% \begin{Synopsis}
% #include <lyric/Thread>

% void Thread::testcancel ();
% \end{Synopsis}
% \end{mansec}

% \begin{mansec}{Description}
%   \begin{manfunc}{Thread::testcancel ()}
%     Tests for pending cancellation and executes it.
%   \end{manfunc}
% \end{mansec}

% \index{Thread::testcancel|)}

% %%=============================================================================

% \newpage
% \index{Thread::yield|(}

% \section{Thread::yield}
% \label{sec:Thread::yield}

% A thread can relinquish the processor voluntarily without blocking by
% calling \code{Thread::yield()}. The thread will then be moved to the
% end of the queue for its static priority and a new thread gets to run.

% \nsThreadNote{yield}{function}

% \begin{mansec}{Synopsis}
% \begin{Synopsis}
% #include <lyric/Thread>

% void Thread::yield ();
% \end{Synopsis}
% \end{mansec}

% \begin{mansec}{Description}
%   \begin{manfunc}{void Thread::yield ()}
%     The calling thread relinquishes the processor, giving another
%     thread a higher chance to run.\\
%     If the current thread is the only thread in the highest priority
%     list at that time, this thread will continue to run after a call
%     to \code{Thread::yield()}.
%   \end{manfunc}
% \end{mansec}

% \index{Thread::yield|)}

%%=============================================================================

%% -*- LaTeX -*-

%%=============================================================================

\begin{manpage}{Suspension}
\let\savedmanlayout=\manlayout
\renewcommand{\manlayout}{vcompact}

A thread can relinquish the processor voluntarily.
\\
It can call \code{yield} to relinquish the processor without blocking.
The thread is moved to the end of the queue for its static priority and a new thread gets to run.
\\
It can call \code{sleep} to relinquish the processor for a given duration.
The thread remains unscheduled for the duration of the sleep.

\mansection{Synopsis}
\begin{mansynopsis}
#include <lyric/self>

void yield ();

void sleep (const Duration& duration);
\end{mansynopsis}

\mansection{Description}
\begin{mandescription}
  \function{void}{yield}{}
    The calling thread relinquishes the processor, giving another thread a higher chance to run.
    If the current thread is the only thread in the highest priority list at that time, this thread will continue to run after a call to \code{yield()}.

  \function{void}{sleep}{const Duration\& \farg{duration}}
    This function causes the calling thread to be suspended from execution until the specified \farg{duration} has elapsed.
    The suspension time may be longer than requested due to scheduling of other activity by the system.
    The sleeping thread does not consume processing resources.
\end{mandescription}

\renewcommand{\manlayout}{\savedmanlayout}
\end{manpage}

%%=============================================================================

\begin{manpage}{Cancellation}
\let\savedmanlayout=\manlayout
\renewcommand{\manlayout}{vcompact}

Cancellation is the mechanism by which a thread can terminate the execution of another thread. 
Functions to \code{set} and \code{get} the thread's cancellation state.
A thread's cancellation state tells how the thread will honor cancellation requests.
Table~\ref{tab:Cancellation} lists the possible thread cancellation states and describes how a thread with a given cancellation descriptor will act on cancellation requests.
\\
\lisle implements a synchronous cancellation scheme only.
This means that a thread with enabled cancellation still needs to perform \code{testcancel} calls from time to time to actually honor a cancellation request.
\\
See also section~\manref{sec:Thrid}, \code{Thrid::cancel()}, for a description on sending a cancellation request to a thread.

\begin{table}[htbp]
  \centering
  \begin{tabular}{|l|l|}
    \hline
    \textbf{Value} & \textbf{Description} \\
    \hline
    \code{cancel::enable} &
    The thread will honor cancellation requests.
    \\
    \hline
    \code{cancel::disable} &
    The thread will ignore cancellation requests.
    \\
    \hline
  \end{tabular}
  \caption{Cancellation states.}
  \label{tab:Cancellation}
\end{table}

\mansection{Synopsis}
\begin{mansynopsis}
#include <lisle/self>

void set (const cancel::State& cancellation);
void get (cancel::State& cancellation);
void testcancel ();
\end{mansynopsis}

\mansection{Description}
\begin{mandescription}
  \function{void}{set}{const cancel::State\& \farg{cancellation}}
  Sets the calling thread's \farg{cancellation} state.

  \function{void}{get}{cancel::State\& \farg{cancellation}}
  Gets the calling thread's \farg{cancellation} state.
  
  \function{void}{testcancel}{}
  Tests for a pending cancellation request and eventually executes it by throwing a \code{thrcancel} exception.
  The purpose if this function is to introduce explicit cancellation checks in long sequences of code that do not call a cancellation point function otherwise.
  \\
  Throwing an exception ensures proper stack unwinding and instances destruction.
  It is important to let exception \code{thrcancel} flow through without being caught.
  See section~\manref{sec:Exceptions.Internal} for details on best practice with \lisle internal exceptions.
\end{mandescription}

\renewcommand{\manlayout}{\savedmanlayout}
\end{manpage}

%%=============================================================================

\begin{manpage}{Termination}
\let\savedmanlayout=\manlayout
\renewcommand{\manlayout}{vcompact}

A thread can terminate by calling the \code{exit} function.
An optional return value can be given in a handle by the terminating thread for an eventual joining thread.
A joining thread gets a \code{terminated} value as return of the \code{join} function.
\\
To guarantee proper stack unwinding and instances destruction \lisle implements the explicit thread termination by throwing a \code{threxit} exception.
It is important to let exception \code{threxit} flow through without being caught.
See section~\manref{sec:Exceptions.Internal} for details on best practice with \lisle internal exceptions.

\mansection{Synopsis}
\begin{mansynopsis}
#include <lyric/self>

void exit ();

template <typename TRT>
void exit (const Handle<TRT>& hretd);
\end{mansynopsis}

\mansection{Description}
\begin{mandescription}
  \function{void}{exit}{}
    This function causes the calling thread to terminate.
    A joining thread gets a \code{terminated} code.

  \function{void}{exit}{const Handle<TRT>\& \farg{hretd}}
    This function causes the calling thread to terminate and return some data stored in \farg{hretd} to a joining thread.
    A joining thread gets a \code{terminated} code.
\end{mandescription}

\renewcommand{\manlayout}{\savedmanlayout}
\end{manpage}

%%=============================================================================

%% -*- LaTeX -*-

\begin{classpage}{Thread}

The \class{Thread} class is a thread startup descriptor. 
Objects of this class are used as first argument to the \code{create} function. 
The class is a container which stores:

\begin{itemize}
  \item The address of the thread's ``main'' function. 
    The function that will run concurrently with the calling thread.
  \item The handle to arguments to pass to the thread on startup, if any.
  \item The thread's startup mode. 
    This mode describes whether the thread starts in suspended mode or in running mode. 
    The default startup mode is the running mode. 
    Table~\ref{tab:Thread::Mode} describes the thread startup modes and their effect on a freshly created thread.
  \item The thread's starting state.
    This state describes whether the thread starts in joinable state or in detached state.
    The default starting state is the joinable state.
    Table~\ref{tab:Thread::State} describes the thread startup states.
\end{itemize}

\begin{table}[htbp]
  \centering
  \begin{tabular}{|l|p{0.7\linewidth}|}
    \hline
    \textbf{Mode} & \textbf{Description} \\
    \hline
    \code{Running} &
    The thread calls its main function as soon as it is created. This
    is the default thread start mode.
    \\
    \hline
    \code{Suspended} &
    The thread suspends itself as soon as it is created, before
    calling its main function. A call to \code{Thrid::resume()}
    is needed to switch the thread back into running state.
    \\
    \hline
  \end{tabular}
  \caption{Thread modes.}
  \label{tab:Thread::Mode}
\end{table}

\begin{table}[htbp]
  \centering
  \begin{tabular}{|l|p{0.7\linewidth}|}
    \hline
    \textbf{State} & \textbf{Description} \\
    \hline
    \code{Joinable} &
    The thread starts in joinable state.
    This is the default thread start state.
    \\
    \hline
    \code{Detached} &
    The thread starts in detached state.
    \\
    \hline
  \end{tabular}
  \caption{Thread states.}
  \label{tab:Thread::State}
\end{table}

\mansection{Synopsis}
\begin{mansynopsis}
#include <lisle/Thread>

class Thread
{
public:
  enum State { Joinable, Detached };
  enum Mode { Running, Suspended };
  typedef void(* Callee)();
  typedef Anondle(* Caller)(void(*)(),Anondle&);
  ~Thread ();
  Thread ();
  Thread (const Thread& that);
  Thread& operator = (const Thread& that);
  Thread (void(* main)(),
          Mode mode = Running, State state = joinable);
  template <typename TAT>
  Thread (void(* main)(Handle<TAT>), const Handle<TAT>& args,
          Mode mode = Running, State state = Joinable);
  template <typename TRT>
  Thread (Handle<TRT>(* main)(),
          Mode mode = Running, State state = Joinable);
  template <typename TRT, typename TAT>
  Thread (Handle<TRT>(* main)(Handle<TAT>), const Handle<TAT>& args,
          Mode mode = Running, State state = Joinable);
  Caller call () const;  
  Callee main () const;
  Anondle& args () const;
  State state () const;
  Mode mode () const;
};
\end{mansynopsis}

\mansection{Description}
\begin{mandescription}
  \destructor
  Destroys \farg{this} thread startup descriptor. Releases all used
  resources.

  \constructor{}
  Constructs \farg{this} thread startup descriptor, setting the
  ``main'' function address to \code{NULL}, and the startup mode to
  \code{Running}.

  \constructor{const Thread\& \farg{that}}
  Constructs \farg{this} thread startup descriptor from \farg{that}
  thread descriptor. All values stored in that
  thread descriptor are cloned into \farg{this} thread
  startup descriptor.

  \operator{Thread\&}{=}{=}{const Thread\& \farg{that}}
  Assigns \farg{this} thread startup descriptor from \farg{that} thread
  descriptor. All values stored in \farg{that}
  thread descriptor are cloned into \farg{this} thread
  startup descriptor. A reference to \farg{this} thread startup
  descriptor is returned for assignment chaining.

\end{mandescription}

\end{classpage}

%% -*- LaTeX -*-

\begin{classpage}{Thrid}

Class \code{Thrid} is a thread identifying structure with thread manipulation
member functions. The thread member functions act on the thread
identified by the thread identifier, not the calling thread. On some
systems thread identifiers are called thread handlers. In \lisle the
terms ``thread identifier'' and ``thread handler'' have the same
meaning and can be exchanged (however, there is no dedicated class).

In order to get a thread id do one of the following:
\begin{itemize}
\item Instantiate a \code{Thrid} object. The default constructor
  will initialize the resulting thread id to identify the calling
  thread.
\item Call the \code{create} function to create a new thread
  of execution. The function returns the thread id of the created
  thread. See section~\manref{sec:Creation} for details about
  thread creation.
\end{itemize}

\mansection{Synopsis}
\begin{mansynopsis}
#include <lisle/Thrid>

class Thrid
{
public:
  ~Thrid ();
  Thrid ();
  Thrid (const Thrid& that);
  Thrid& operator = (const Thrid& that);
  operator uint32_t () const;
  bool operator == (const Thrid& id) const;
  bool operator != (const Thrid& id) const;
  void detach ()
    throw (detach);
  void cancel ()
    throw ();
  void suspend ()
    throw (deadlock);
  void resume ()
    throw ();
  Exit join ();
    throw (join, deadlock, thrcancel);
  template <typename TRT>
  Exit join (Handle<TRT>& hretd);
    throw (join, deadlock, thrcancel);
};
\end{mansynopsis}

\mansection{Description}
\begin{mandescription}
  \destructor
  Destroys \farg{this} thread id, releasing all used resources. The
  thread itself is not affected. Only the thread id is destroyed.

  \constructor{}
  Constructs \farg{this} thread id.\\
  The result identifies the calling thread.

  \constructor{const Thrid\& \farg{that}}
  Constructs \farg{this} from \farg{that} thread id.  The data
  stored in \farg{that} thread id is cloned into \farg{this} thread id.

  \operator{Thrid\&}{=}{=}{const Thread::Id\& \farg{that}}
  Assigns \farg{this} from \farg{that} thread id. The data
  stored in \farg{that} thread id is cloned into \farg{this} thread id.

  \operator[const]{}{uint32\_t}{uint32}{}
  Casts \farg{this} thread id into a 32 bit unsigned integer. This
  cast can be used to print out \farg{this} thread id in numeric
  format. Most debuggers show the running threads by a number. This
  cast returns the same number for the thread identified by
  \farg{this} thread id.

  \operator[const]{bool}{==}{==}{const Thrid\& \farg{that}}
  Compares \farg{this} with \farg{that} rvalue thread id.
  Returns \code{true} if both ids identify the same thread,
  \code{false} if not.

  \operator[const]{bool}{!=}{"!=}{const Thrid\& \farg{that}}
  Compares \farg{this} with \farg{that} rvalue thread id. Returns
  \code{true} if both ids identify different threads, \code{false}
  if not.

  \function{Exit}{join}{Handle<TRT>\& hretd}
  Suspends the execution of the calling thread until \farg{this} thread terminates, either by thread termination or cancellation.
  \\
  The returned \code{Exit} stores the mean by which \farg{this} thread terminated:
  \begin{description}
    \item[\code{terminated}] if \farg{this} thread terminated normally or
      if \farg{this} thread was exited with a call to \code{exit},
    \item[\code{canceled}] if \farg{this} thread was canceled.
    \item[\code{exceptioned}] if an uncaught exception was thrown in \farg{this} thread.
  \end{description}
  The optional \farg{hretd} allows the calling thread to retrieve a handle to data eventually returned by \farg{this} thread when it terminated.
  See section~\manref{sec:Termination} for details about the \code{exit} function and how it allows a thread to return data to a joining thread.
  \\
  The thread with \farg{this} id must be in the joinable state: it must not have been detached calling \code{detach} or created in detached state.
  If \farg{this} thread is not in joinable state a \class{join} exception with reason \code{join::detached} is thrown.
  \\
  When a joinable thread terminates, its memory resources (thread descriptor and stack) are not released until another thread performs a \code{join} on it.
  Therefore, a joinable thread {\bf must} be joined in order to avoid memory leaks.
  \\
  At most one thread can join \farg{this} thread. 
  If another thread is already waiting for termination of \farg{this} thread a \class{join} exception with reason \code{join::duplicate} is thrown.
  \begin{exception}
    \item[join] is thrown if joining \farg{this} thread failed.\\
      The exception's reason of failure can be:
      \begin{exreason}
        \item[join::missing] if \farg{this} thread no longer exists.
        \item[join::detached] if \farg{this} thread is in detached state.
        \item[join::duplicate] if another thread is already joining \farg{this} thread.
      \end{exreason}
    \item[deadlock] is thrown if \farg{this} thread is the calling thread.
      Self joining would deadlock \farg{this} thread.
  \end{exception}

  \function{void}{detach}{}
  Puts \farg{this} thread in detached state.
  This guarantees that the memory resources consumed by \farg{this} thread will be released immediately when \farg{this} thread terminates. 
  However, this prevents other threads from synchronizing on the termination of \farg{this} thread by joining it.
  \\
  A thread can be created initially in the detached state, by setting the \code{Thread::State} to \code{Thread::Detached}.
  In contrast method \code{detach} applies to threads created in the joinable state, and which need to be put in detached state later.
  \\
  After \code{detach} completes, subsequent attempts to join \farg{this} thread will fail with an exception.
  \\
  If another thread is already joining \farg{this} thread at the time \code{detach} is called, the call does nothing and leaves \farg{this} thread in the joinable state.
  \begin{exception}
    \item[detach] is thrown if detaching \farg{this} thread failed.
      \\
      The exception's reason of failure can be:
      \begin{exreason}
        \item[detach::missing] if \farg{this} thread no longer exists.
        \item[detach::detached] if \farg{this} thread is already in the detached state.
      \end{exreason}
  \end{exception}

  \function{void}{cancel}{}
  Depending on its settings, the target thread can then either ignore the request, or defer it till it reaches a cancellation point.
  \\
  When a thread eventually honors a cancellation request, it performs as if \code{exit} has been called at that point. 
  See section~\manref{sec:Cancellation} for more information on thread cancellation.

  \function{void}{suspend}{}
  Suspends execution of \farg{this} thread.
  \\
  A suspended thread will not be awakened by a signal. 
  The signal stays pending until the execution of \farg{this} thread is resumed.
  \begin{exception}
    \item[deadlock] is thrown if suspending
      \farg{this} thread would deadlock the process.
  \end{exception}

  \function{void}{resume}{}
  Resumes the execution of \farg{this} suspended thread.
\end{mandescription}

\mansection{Cancellation} 
\code{Thrid::join()} is a cancellation point. 
If a thread is canceled while suspended in a call to \code{Thrid::join()}, the thread execution resumes immediately and the cancellation is executed without waiting for \farg{this} thread to terminate. 
If cancellation occurs during \code{Thrid::join()}, the thread to be joined remains not joined.

\end{classpage}

%% -*- LaTeX -*-

\begin{classpage}{Once}

The \class{Once} is a base class for {\em once controls}.  The
purpose of a once control is to ensure that a piece of code is
executed at most once, whatever number of calls to its \code{run}
function, in whichever thread. Once controls can be used for thread
safe initialization for example.

The first time the \code{run} function of a once control is called,
the dervied \code{main} function is called without arguments and the
once control records that the call has been performed. Each subsequent
call to the \code{run} function of the same once control does nothing.

A once control must be a \code{Once} derived class where the
pure virtual \code{main} function is overloaded. The \code{main}
function is the place where the actions to perform once are listed.
The \code{run} function only checks that \code{main} will be run
exactly once, thread safely. The \code{run} function should not be
overloaded, or only with care to call the base class' \code{run}.

\mansection{Synopsis}
\begin{mansynopsis}
#include <lisle/Once>

class Once
{
public:
  virtual ~Once ();
  Once ();
  void run ();
  virtual void main () = 0;
};
\end{mansynopsis}

\mansection{Description}
\begin{mandescription}
  \destructor
  Destroys \farg{this} once control. Releases all resources used by
  \farg{this} control.

  \constructor{}
  Constructs \farg{this} once control.

  \function{void}{run}{}
  Runs \farg{this} once control. Checks thread safely that it is the
  first call and calls the overloaded \code{main} function. If
  \farg{this} once control's \code{run} function has already been
  called once, the function returns immediately.

  \function{void}{main}{}
  This is the pure virtual function that must be overloaded. It is
  the place where the actions to be performed once for \farg{this}
  control must the implemented.
\end{mandescription}

\end{classpage}

%% -*- LaTeX -*-

\begin{classpage}{Local}

Programs sometimes need global or static variables that have different
values in different threads. Since threads share one memory space,
this cannot be achieved with regular variables. The
\class{Local} class is the \lisle threads answer to this need.

Each thread possesses a private memory block, the thread-specific data
area, or TSD area for short. This area is accessible with objects of
the \class{Local} class. Each instance of the
\class{Local} class is common to all threads, but the value
associated with a given \code{Local} can be different in each
thread.

\mansection{Synopsis}
\begin{mansynopsis}
#include <lisle/Local>

template <typename T>
class Local
{
public:
  ~Local ();
  Local ()
    throw (resource);
  T& operator = (const T& data)
    throw (resource);
  operator T* ()
    throw ();
  operator T* () const
    throw ();
  T* operator -> ()
    throw ();
  T* operator -> () const
    throw ();
private:
  // @textnit$Disable cloning?
  Local (const Local&);
  Local& operator = (const Local&);
};
\end{mansynopsis}

\mansection{Description}
\begin{mandescription}
  \destructor
  Destroys \farg{this} TSD, releasing all used resources.

  \constructor{}
  Constructs \farg{this} TSD. Notice the abscence of a constructor
  with default argument. It is not possible to construct a TSD with an
  initial value other than \code{NULL}. Each thread \emph{must}
  initialize its ``instance'' of \farg{this} TSD with the assignment
  operator (see below).
  \begin{exception}
    \item[resource] is thrown if the TSD area space is
      exhausted.
  \end{exception}

  \operator{T\&}{=}{=}{const T\& data}
  Assigns some \farg{data} to \farg{this} TSD. Returns a reference to
  the stored data for assignment operation chaining.
  
  This operator uses \code{T::operator =} to store \farg{data} in
  \farg{this} TSD. Depending on the implementation of the \code{=}
  operator for a given \code{T} class, especially for containers,
  what will finally be stored in \farg{this} TSD may vary (data or
  pointer).
  
  Assigning data to \farg{this} TSD is \emph{mandatory} before using
  data stored in \farg{this} TSD. At least one assignment has to be
  done before \farg{this} TSD can be used with the casting operators
  described below.  As analogy consider \farg{this} TSD as a pointer,
  set to \code{NULL}. As long as a pointer is not initialized with
  \code{new}, \code{malloc} or assignment to an initialized pointer it
  can't be used without the risk of memory leaks. It is the same for
  TSDs. The only difference is that only this assigment operator is
  available for initializing \farg{this} TSD. Note that the
  initialization (aka assignment) \emph{must} be done once in
  \emph{each} thread.

  \begin{exception}
    \item[resource] is thrown if there was not enough
      memory to store one item of type \code{T} in the calling
      thread's memory space (the shared memory space, not the TSD
      area).
  \end{exception}

  \operator{}{T*}{T*}{}
  Returns the pointer to the data stored in \farg{this} TSD.
  \operator[const]{}{T*}{const T*}{}
  Returns the pointer to the data stored in \farg{this} TSD.

  \operator{T*}{->}{-$>$}{}
  Returns the pointer to the data stored in \farg{this} TSD, giving
  access to data and function members if \code{T} is a structure or a
  class.
  \operator[const]{T*}{->}{-$>$}{}
  Returns the pointer to the data stored in \farg{this} TSD, giving
  access to data and function members if \code{T} is a structure or a
  class.

\end{mandescription}

\end{classpage}

%------------------------------------------------------------------------------

%==============================================================================
\chapter{Miscellaneous}
\label{cha:Miscellaneous}

\manminitoc
\noindent
Helper classes.

%% -*- LaTeX -*-

\begin{classpage}{Handle}

A \class{Handle} is a thread safe smart pointer.
Thread safeness is achieved by means of an atomic counter.
As with other smart pointers, a handle releases data once the last handle referencing that data is destroyed.

\mansection{Synopsis}
\begin{mansynopsis}
#include <lisle/Handle>

template <typename T>
class Handle
{
public:
  ~Handle ();
  Handle (T* data = 0);
  Handle (const Handle& that);
  Handle& operator = (const Handle& that)
  operator T* ();
  operator T* () const;
  T* operator -> ();
  T* operator -> ();
};
\end{mansynopsis}

\mansection{Description}
\begin{mandescription}
  \destructor
  Destroys \farg{this} handle,
  atomically decreasing the shared usage counter,
  and eventually releasing used resources when the usage counter reaches zero.

  \constructor{T* data}
  Constructs \farg{this} handle referencing the given \farg{data}.
  If no \farg{data} is specified a handle referencing \code{NULL} is created.
  
  \constructor{const Handle\& that}
  Constructs \farg{this} handle from \farg{that} handle.
  Both handles reference the same data,
  and the shared usage counter is increased atomically.

  \operator{Handle\&}{=}{=}{const Handle\& that}
  Assigns \farg{that} handle to \farg{this} handle.
  If both handles are the same then the operation has no effect.
  Otherwise \farg{this} handle is first ``cleaned'' (dereferenced, with eventual data release),
  then \farg{that} handle is cloned into \farg{this} handle,
  and the shared usage counter is increased atomically.

  \operator{}{T*}{T*}{}
  Returns the pointer to the data stored in \farg{this} handle.
  \operator[const]{}{T*}{const T*}{}
  Returns the pointer to the data stored in \farg{this} handle.

  \operator{T*}{->}{-$>$}{}
  Returns the pointer to the data stored in \farg{this} handle, giving
  access to data and function members if \code{T} is a structure or a
  class.
  \operator[const]{T*}{->}{-$>$}{}
  Returns the pointer to the data stored in \farg{this} handle, giving
  access to data and function members if \code{T} is a structure or a
  class.

\end{mandescription}

\end{classpage}

%% -*- LaTeX -*-

\begin{classpage}{Anondle}

An \class{Anondle} is an anonymous handle.
It allows to remove type information from a handle a bit in the same way as \code{void*} removes type information from C pointers.

\mansection{Synopsis}
\begin{mansynopsis}
#include <lisle/Anondle>

class Anondle
{
public:
  ~Anondle ();
  Anondle ();
  Anondle (const Anondle& that);
  Anondle& operator = (const Anondle& that);
  template <typename T> Anondle (const Handle<T>& handle);
  template <typename T> Handle<T>* handle () const;
};
\end{mansynopsis}

\mansection{Description}
\begin{mandescription}
  \destructor
  Destroys \farg{this} anonymous handle.
  
  \constructor{}
  Constructs \farg{this} empty anonymous handle.
  
  \constructor{const Anondle\& \farg{that}}
  Constructs \farg{this} anonymous handle from \farg{that} anonymous handle.

  \operator{Handle\&}{=}{=}{const Anondle\& \farg{that}}
  Assigns \farg{that} anonymous handle to \farg{this} anonymous handle.
  
  \constructor{const Handle<T>\& \farg{handle}}
  Constructs \farg{this} anonymous handle from \farg{handle}.
  
  \function[const]{Handle<T>*}{handle}{}
  Returns a pointer to the handle stored in \farg{this} anonymous handle.
\end{mandescription}

\end{classpage}

%% -*- LaTeX -*-

\begin{classpage}{Countic}

A \class{Countic} is a thread safe atomic counter.
A \class{Countic} can be atomically incremented and tested for zero or atomically decremented and tested for zero.

\mansection{Synopsis}
\begin{mansynopsis}
#include <lisle/Countic>

class Countic
{
public:
  Countic (int32_t val = 0);
  operator int32_t () const;
  void inc ();
  void dec ();
  bool inctz ();
  bool dectz ();
private:
  // @textnit$Disable cloning?
  Countic (const Countic&);
  Countic& operator = (const Countic&);
};
\end{mansynopsis}

\mansection{Description}
\begin{mandescription}
  \constructor{int32\_t \farg{val}}
  Constructs \farg{this} countic, assigning it the optional \farg{val}ue.
  If no \farg{val}ue is specified a countic with value zero is created.
  
  \operator[const]{}{int32\_t}{int32\_t}{}
  Returns the value stored in \farg{this} countic.
  
  \function{void}{inc}{}
  Increments \farg{this} countic by one.
  
  \function{void}{dec}{}
  Decrements \farg{this} countic by one.
  
  \function{bool}{inctz}{}
  Atomically increments and tests for zero \farg{this} countic.
  Returns \code{true} it \farg{this} countic falls to zero.
  
  \function{bool}{dectz}{}
  Atomically decrements and tests for zero \farg{this} countic.
  Returns \code{true} it \farg{this} countic falls to zero.
\end{mandescription}

\end{classpage}

%% -*- LaTeX -*-

\begin{classpage}{Duration}

A class to store time spans.

\mansection{Synopsis}
\begin{mansynopsis}
#include <lisle/Duration>

class Duration
{
public:
  ~Duration ();
  Duration ();
  Duration (const Duration& that);
  Duration& operator = (const Duration& that);
  Duration (double seconds);
  int64_t sec () const;
  uint32_t nsec () const;
  Duration operator - () const;
  bool operator == (const Duration& that) const;
  bool operator != (const Duration& that) const;
  bool operator < (const Duration& that) const;
  bool operator > (const Duration& that) const;
  bool operator <= (const Duration& that) const;
  bool operator >= (const Duration& that) const;
  Duration& operator += (const Duration& that);
  Duration& operator -= (const Duration& that);
};
Duration operator + (const Duration& lho, const Duration& rho);
Duration operator - (const Duration& lho, const Duration& rho);
\end{mansynopsis}

\mansection{Description}
\begin{mandescription}
  \destructor
  Destroys \farg{this} duration.
  
  \constructor{}
  Constructs \farg{this} duration as a zero time span.
  
  \constructor{const Duration\& \farg{that}}
  Constructs \farg{this} duration from \farg{that} duration.
  The time span stored in \farg{that} is copied into \farg{this}.
  
  \operator{Duration\&}{=}{=}{const Duration\& \farg{that}}
  Assigns \farg{that} duration to \farg{this} duration.
  The time span stored in \farg{that} is copied into \farg{this}.
  A reference to \farg{this} duration is returned for assignment chaining.
  
  \constructor{double \farg{seconds}}
  Constructs \farg{this} duration for a time span of the given \farg{seconds}.
  A \farg{seconds} value of \code{3.25} means 3 seconds and 25 hundred's of a second.
  
  \function[const]{int64\_t}{sec}{}
  Returns the number of seconds of \farg{this} time span.
  
  \function[const]{uint32\_t}{nsec}{}
  Returns the number of nano-seconds of \farg{this} time span, e.g. the decimal part, rounded to the nano-second.
  
  \operator[const]{Duration}{-}{-}{}
  Returns the negative of \farg{this} duration.
  
  \operator[const]{bool}{==}{==}{const Duration\& \farg{that}}
  Returns \code{true} if \farg{this} duration stores the same time span as \farg{that} duration, \code{false} if not.

  \operator[const]{bool}{!=}{"!=}{const Duration\& \farg{that}}
  Returns \code{true} if \farg{this} duration stores a different time span as \farg{that} duration, \code{false} if not.
  
  \operator[const]{bool}{<}{<}{const Duration\& \farg{that}}
  Returns \code{true} if \farg{this} duration stores a time span smaller than \farg{that} duration, \code{false} if not.

  \operator[const]{bool}{>}{>}{const Duration\& \farg{that}}
  Returns \code{true} if \farg{this} duration stores a time span greater than \farg{that} duration, \code{false} if not.

  \operator[const]{bool}{<=}{<=}{const Duration\& \farg{that}}
  Returns \code{true} if \farg{this} duration stores a time span smaller or equal than \farg{that} duration, \code{false} if not.

  \operator[const]{bool}{>=}{>=}{const Duration\& \farg{that}}
  Returns \code{true} if \farg{this} duration stores a time span greater or equal than \farg{that} duration, \code{false} if not.
  
  \operator{Duration\&}{+=}{+=}{const Duration\& \farg{that}}
  Adds \farg{that} time span to \farg{this} duration and returns a reference to \farg{this} for operation chaining.
  
  \operator{Duration\&}{-=}{-=}{const Duration\& \farg{that}}
  Substracts \farg{that} time span from \farg{this} duration and returns a reference to \farg{this} for operation chaining.
  
  \operator{Duration}{+}{+}{const Duration\& \farg{lho}, const Duration\& \farg{rho}}
  Returns the time span sum of \farg{lho} and \farg{rho} durations.

  \operator{Duration}{-}{-}{const Duration\& \farg{lho}, const Duration\& \farg{rho}}
  Returns the time span difference of \farg{lho} and \farg{rho} durations.
\end{mandescription}

\end{classpage}

%------------------------------------------------------------------------------

%==============================================================================
\chapter{Exceptions}
\label{cha:Exceptions}

\manminitoc
\noindent

\section{Resource}
\label{sec:Exceptions.Resource}
A \class{resource} exception is thrown whenever there are not enough system resources left to create an instance of a thread related item.
Expect a \class{resource} exception to be thrown by a constructor.

\section{Permission}
\label{sec:Exceptions.Permission}
A \class{permission} exception is thrown whenever the program doesn't have permission to perform a given operation.
Expect a \class{permission} exception to be thrown by a destructor, or by a setting on a thread the operating system wouldn't allow.

\section{Deadlock}
\label{sec:Exceptions.Deadlock}
Expect a \class{deadlock} exception when \lisle detects an operation that would lock a thread in a state from which it can't exit.
For example self joining a thread would deadlock.
An attempt to relock a mutex will also throw a \class{deadlock} exception.

\section{Overflow}
\label{sec:Exceptions.Overflow}
Expect an \class{overflow} exception when a post operation would overflow a semaphore count.

\section{Timeout}
\label{sec:Exceptions.Timeout}
A \class{timeout} exception is thrown whenever a timed wait terminates on the duration condition.
Expect a \class{timeout} exception on all timed waits.

\section{Virthread}
\label{sec:Exceptions.Virthread}
A \class{virthread} exception is thrown whenever a null function thread creation is attempted.

\section{Internal}
\label{sec:Exceptions.Internal}
\lisle uses exceptions for thread termination and cancellation.
Using exceptions ensures that the stack is unwound correctly and that all used resources are released properly.
Internal exceptions all derive from the \code{threption} base class.
It is mandatory to let \code{threption} exceptions flow freely without being caught.
\\
Thus a
\\
\indent \code{catch~(...)}
\\
clause must be preceded by a
\\
\indent \code{catch~(threption\&)~\{throw;\}}
\\
clause for thread cancellation and termination to work.
The correct catch all is thus:
\\
\indent \code{catch~(threption\&)~\{throw;\}}
\\
\indent \code{catch~(...)}
%------------------------------------------------------------------------------

%==============================================================================
\appendix

\newpage
\chapter{Terms and Conditions}
\label{app:Terms}

\section{Library Usage}
\label{app:LGPL}
The \lisle thread C++ library is distributed under the terms and
conditions of the GNU library general public license. The following
text describes the terms and conditions for copying, distribution and
modification of \lisle. In the following text ``library'' refers to
the ``\lisle C++ thread library''.
\vspace*{5ex}
\begin{center}
  GNU LESSER GENERAL PUBLIC LICENSE\\
  TERMS AND CONDITIONS FOR COPYING, DISTRIBUTION AND MODIFICATION
\end{center}

\vspace*{2ex}

\begin{enumerate}
\setcounter{enumi}{-1}
\item
  This License Agreement applies to any software library or other
program which contains a notice placed by the copyright holder or
other authorized party saying it may be distributed under the terms of
this Lesser General Public License (also called "this License").
Each licensee is addressed as "you".

  A "library" means a collection of software functions and/or data
prepared so as to be conveniently linked with application programs
(which use some of those functions and data) to form executables.

  The "Library", below, refers to any such software library or work
which has been distributed under these terms.  A "work based on the
Library" means either the Library or any derivative work under
copyright law: that is to say, a work containing the Library or a
portion of it, either verbatim or with modifications and/or translated
straightforwardly into another language.  (Hereinafter, translation is
included without limitation in the term "modification".)

  "Source code" for a work means the preferred form of the work for
making modifications to it.  For a library, complete source code means
all the source code for all modules it contains, plus any associated
interface definition files, plus the scripts used to control compilation
and installation of the library.

  Activities other than copying, distribution and modification are not
covered by this License; they are outside its scope.  The act of
running a program using the Library is not restricted, and output from
such a program is covered only if its contents constitute a work based
on the Library (independent of the use of the Library in a tool for
writing it).  Whether that is true depends on what the Library does
and what the program that uses the Library does.
  
\item
  You may copy and distribute verbatim copies of the Library's
complete source code as you receive it, in any medium, provided that
you conspicuously and appropriately publish on each copy an
appropriate copyright notice and disclaimer of warranty; keep intact
all the notices that refer to this License and to the absence of any
warranty; and distribute a copy of this License along with the
Library.

  You may charge a fee for the physical act of transferring a copy,
and you may at your option offer warranty protection in exchange for a
fee.

\item
  You may modify your copy or copies of the Library or any portion
of it, thus forming a work based on the Library, and copy and
distribute such modifications or work under the terms of Section 1
above, provided that you also meet all of these conditions:

\begin{enumerate}
  \item
    The modified work must itself be a software library.

  \item
    You must cause the files modified to carry prominent notices
    stating that you changed the files and the date of any change.

  \item
    You must cause the whole of the work to be licensed at no
    charge to all third parties under the terms of this License.

  \item
    If a facility in the modified Library refers to a function or a
    table of data to be supplied by an application program that uses
    the facility, other than as an argument passed when the facility
    is invoked, then you must make a good faith effort to ensure that,
    in the event an application does not supply such function or
    table, the facility still operates, and performs whatever part of
    its purpose remains meaningful.

    (For example, a function in a library to compute square roots has
    a purpose that is entirely well-defined independent of the
    application.  Therefore, Subsection 2d requires that any
    application-supplied function or table used by this function must
    be optional: if the application does not supply it, the square
    root function must still compute square roots.)
\end{enumerate}

These requirements apply to the modified work as a whole.  If
identifiable sections of that work are not derived from the Library,
and can be reasonably considered independent and separate works in
themselves, then this License, and its terms, do not apply to those
sections when you distribute them as separate works.  But when you
distribute the same sections as part of a whole which is a work based
on the Library, the distribution of the whole must be on the terms of
this License, whose permissions for other licensees extend to the
entire whole, and thus to each and every part regardless of who wrote
it.

Thus, it is not the intent of this section to claim rights or contest
your rights to work written entirely by you; rather, the intent is to
exercise the right to control the distribution of derivative or
collective works based on the Library.

In addition, mere aggregation of another work not based on the Library
with the Library (or with a work based on the Library) on a volume of
a storage or distribution medium does not bring the other work under
the scope of this License.

\item
  You may opt to apply the terms of the ordinary GNU General Public
License instead of this License to a given copy of the Library.  To do
this, you must alter all the notices that refer to this License, so
that they refer to the ordinary GNU General Public License, version 2,
instead of to this License.  (If a newer version than version 2 of the
ordinary GNU General Public License has appeared, then you can specify
that version instead if you wish.)  Do not make any other change in
these notices.

  Once this change is made in a given copy, it is irreversible for
that copy, so the ordinary GNU General Public License applies to all
subsequent copies and derivative works made from that copy.

  This option is useful when you wish to copy part of the code of
the Library into a program that is not a library.

\item
  You may copy and distribute the Library (or a portion or
derivative of it, under Section 2) in object code or executable form
under the terms of Sections 1 and 2 above provided that you accompany
it with the complete corresponding machine-readable source code, which
must be distributed under the terms of Sections 1 and 2 above on a
medium customarily used for software interchange.

  If distribution of object code is made by offering access to copy
from a designated place, then offering equivalent access to copy the
source code from the same place satisfies the requirement to
distribute the source code, even though third parties are not
compelled to copy the source along with the object code.

\item
  A program that contains no derivative of any portion of the
Library, but is designed to work with the Library by being compiled or
linked with it, is called a "work that uses the Library".  Such a
work, in isolation, is not a derivative work of the Library, and
therefore falls outside the scope of this License.

  However, linking a "work that uses the Library" with the Library
creates an executable that is a derivative of the Library (because it
contains portions of the Library), rather than a "work that uses the
library".  The executable is therefore covered by this License.
Section 6 states terms for distribution of such executables.

  When a "work that uses the Library" uses material from a header file
that is part of the Library, the object code for the work may be a
derivative work of the Library even though the source code is not.
Whether this is true is especially significant if the work can be
linked without the Library, or if the work is itself a library.  The
threshold for this to be true is not precisely defined by law.

  If such an object file uses only numerical parameters, data
structure layouts and accessors, and small macros and small inline
functions (ten lines or less in length), then the use of the object
file is unrestricted, regardless of whether it is legally a derivative
work.  (Executables containing this object code plus portions of the
Library will still fall under Section 6.)

  Otherwise, if the work is a derivative of the Library, you may
distribute the object code for the work under the terms of Section 6.
Any executables containing that work also fall under Section 6,
whether or not they are linked directly with the Library itself.

\item
  As an exception to the Sections above, you may also combine or
link a "work that uses the Library" with the Library to produce a
work containing portions of the Library, and distribute that work
under terms of your choice, provided that the terms permit
modification of the work for the customer's own use and reverse
engineering for debugging such modifications.

  You must give prominent notice with each copy of the work that the
Library is used in it and that the Library and its use are covered by
this License.  You must supply a copy of this License.  If the work
during execution displays copyright notices, you must include the
copyright notice for the Library among them, as well as a reference
directing the user to the copy of this License.  Also, you must do one
of these things:

\begin{enumerate}
  \item
    Accompany the work with the complete corresponding
    machine-readable source code for the Library including whatever
    changes were used in the work (which must be distributed under
    Sections 1 and 2 above); and, if the work is an executable linked
    with the Library, with the complete machine-readable "work that
    uses the Library", as object code and/or source code, so that the
    user can modify the Library and then relink to produce a modified
    executable containing the modified Library.  (It is understood
    that the user who changes the contents of definitions files in the
    Library will not necessarily be able to recompile the application
    to use the modified definitions.)

  \item
    Use a suitable shared library mechanism for linking with the
    Library.  A suitable mechanism is one that (1) uses at run time a
    copy of the library already present on the user's computer system,
    rather than copying library functions into the executable, and (2)
    will operate properly with a modified version of the library, if
    the user installs one, as long as the modified version is
    interface-compatible with the version that the work was made with.

  \item
    Accompany the work with a written offer, valid for at
    least three years, to give the same user the materials
    specified in Subsection 6a, above, for a charge no more
    than the cost of performing this distribution.

  \item
    If distribution of the work is made by offering access to copy
    from a designated place, offer equivalent access to copy the above
    specified materials from the same place.

  \item
    Verify that the user has already received a copy of these
    materials or that you have already sent this user a copy.
\end{enumerate}

  For an executable, the required form of the "work that uses the
Library" must include any data and utility programs needed for
reproducing the executable from it.  However, as a special exception,
the materials to be distributed need not include anything that is
normally distributed (in either source or binary form) with the major
components (compiler, kernel, and so on) of the operating system on
which the executable runs, unless that component itself accompanies
the executable.

  It may happen that this requirement contradicts the license
restrictions of other proprietary libraries that do not normally
accompany the operating system.  Such a contradiction means you cannot
use both them and the Library together in an executable that you
distribute.

\item
  You may place library facilities that are a work based on the
Library side-by-side in a single library together with other library
facilities not covered by this License, and distribute such a combined
library, provided that the separate distribution of the work based on
the Library and of the other library facilities is otherwise
permitted, and provided that you do these two things:

\begin{enumerate}
  \item 
    Accompany the combined library with a copy of the same work
    based on the Library, uncombined with any other library
    facilities.  This must be distributed under the terms of the
    Sections above.

  \item
    Give prominent notice with the combined library of the fact
    that part of it is a work based on the Library, and explaining
    where to find the accompanying uncombined form of the same work.
\end{enumerate}

\item
  You may not copy, modify, sublicense, link with, or distribute
the Library except as expressly provided under this License.  Any
attempt otherwise to copy, modify, sublicense, link with, or
distribute the Library is void, and will automatically terminate your
rights under this License.  However, parties who have received copies,
or rights, from you under this License will not have their licenses
terminated so long as such parties remain in full compliance.

\item
  You are not required to accept this License, since you have not
signed it.  However, nothing else grants you permission to modify or
distribute the Library or its derivative works.  These actions are
prohibited by law if you do not accept this License.  Therefore, by
modifying or distributing the Library (or any work based on the
Library), you indicate your acceptance of this License to do so, and
all its terms and conditions for copying, distributing or modifying
the Library or works based on it.

\item
  Each time you redistribute the Library (or any work based on the
Library), the recipient automatically receives a license from the
original licensor to copy, distribute, link with or modify the Library
subject to these terms and conditions.  You may not impose any further
restrictions on the recipients' exercise of the rights granted herein.
You are not responsible for enforcing compliance by third parties with
this License.

\item
  If, as a consequence of a court judgment or allegation of patent
infringement or for any other reason (not limited to patent issues),
conditions are imposed on you (whether by court order, agreement or
otherwise) that contradict the conditions of this License, they do not
excuse you from the conditions of this License.  If you cannot
distribute so as to satisfy simultaneously your obligations under this
License and any other pertinent obligations, then as a consequence you
may not distribute the Library at all.  For example, if a patent
license would not permit royalty-free redistribution of the Library by
all those who receive copies directly or indirectly through you, then
the only way you could satisfy both it and this License would be to
refrain entirely from distribution of the Library.

If any portion of this section is held invalid or unenforceable under any
particular circumstance, the balance of the section is intended to apply,
and the section as a whole is intended to apply in other circumstances.

It is not the purpose of this section to induce you to infringe any
patents or other property right claims or to contest validity of any
such claims; this section has the sole purpose of protecting the
integrity of the free software distribution system which is
implemented by public license practices.  Many people have made
generous contributions to the wide range of software distributed
through that system in reliance on consistent application of that
system; it is up to the author/donor to decide if he or she is willing
to distribute software through any other system and a licensee cannot
impose that choice.

This section is intended to make thoroughly clear what is believed to
be a consequence of the rest of this License.

\item
  If the distribution and/or use of the Library is restricted in
certain countries either by patents or by copyrighted interfaces, the
original copyright holder who places the Library under this License may add
an explicit geographical distribution limitation excluding those countries,
so that distribution is permitted only in or among countries not thus
excluded.  In such case, this License incorporates the limitation as if
written in the body of this License.

\item
  The Free Software Foundation may publish revised and/or new
versions of the Lesser General Public License from time to time.
Such new versions will be similar in spirit to the present version,
but may differ in detail to address new problems or concerns.

Each version is given a distinguishing version number.  If the Library
specifies a version number of this License which applies to it and
"any later version", you have the option of following the terms and
conditions either of that version or of any later version published by
the Free Software Foundation.  If the Library does not specify a
license version number, you may choose any version ever published by
the Free Software Foundation.

\item
  If you wish to incorporate parts of the Library into other free
programs whose distribution conditions are incompatible with these,
write to the author to ask for permission.  For software which is
copyrighted by the Free Software Foundation, write to the Free
Software Foundation; we sometimes make exceptions for this.  Our
decision will be guided by the two goals of preserving the free status
of all derivatives of our free software and of promoting the sharing
and reuse of software generally.

\vspace*{2ex}
\centerline{NO WARRANTY}

\item
  BECAUSE THE LIBRARY IS LICENSED FREE OF CHARGE, THERE IS NO
WARRANTY FOR THE LIBRARY, TO THE EXTENT PERMITTED BY APPLICABLE LAW.
EXCEPT WHEN OTHERWISE STATED IN WRITING THE COPYRIGHT HOLDERS AND/OR
OTHER PARTIES PROVIDE THE LIBRARY "AS IS" WITHOUT WARRANTY OF ANY
KIND, EITHER EXPRESSED OR IMPLIED, INCLUDING, BUT NOT LIMITED TO, THE
IMPLIED WARRANTIES OF MERCHANTABILITY AND FITNESS FOR A PARTICULAR
PURPOSE.  THE ENTIRE RISK AS TO THE QUALITY AND PERFORMANCE OF THE
LIBRARY IS WITH YOU.  SHOULD THE LIBRARY PROVE DEFECTIVE, YOU ASSUME
THE COST OF ALL NECESSARY SERVICING, REPAIR OR CORRECTION.

\item
  IN NO EVENT UNLESS REQUIRED BY APPLICABLE LAW OR AGREED TO IN
WRITING WILL ANY COPYRIGHT HOLDER, OR ANY OTHER PARTY WHO MAY MODIFY
AND/OR REDISTRIBUTE THE LIBRARY AS PERMITTED ABOVE, BE LIABLE TO YOU
FOR DAMAGES, INCLUDING ANY GENERAL, SPECIAL, INCIDENTAL OR
CONSEQUENTIAL DAMAGES ARISING OUT OF THE USE OR INABILITY TO USE THE
LIBRARY (INCLUDING BUT NOT LIMITED TO LOSS OF DATA OR DATA BEING
RENDERED INACCURATE OR LOSSES SUSTAINED BY YOU OR THIRD PARTIES OR A
FAILURE OF THE LIBRARY TO OPERATE WITH ANY OTHER SOFTWARE), EVEN IF
SUCH HOLDER OR OTHER PARTY HAS BEEN ADVISED OF THE POSSIBILITY OF SUCH
DAMAGES.

\end{enumerate}

\begin{center}
  END OF TERMS AND CONDITIONS
\end{center}


\newpage
\section{Document Usage}
\label{app:FDL}
The \lisle thread C++ library documentation is distributed under the
terms and conditions of the GNU free documentation license. The
following text describes the terms and conditions for copying,
distribution and modification of the \lisle documentation. In the
following text ``document'' refers to the ``\lisle C++ thread library
documentation'' (this document).
%% -*- LaTeX -*-

\subsection{Applicability and Definitions}
\label{applicability}

This License applies to any manual or other work, in any medium, that
contains a notice placed by the copyright holder saying it can be
distributed under the terms of this License.  Such a notice grants a
world-wide, royalty-free license, unlimited in duration, to use that
work under the conditions stated herein.  The ``Document'', below,
refers to any such manual or work.  Any member of the public is a
licensee, and is addressed as ``you''.  You accept the license if you
copy, modify or distribute the work in a way requiring permission
under copyright law.

A ``Modified Version'' of the Document means any work containing the
Document or a portion of it, either copied verbatim, or with
modifications and/or translated into another language.

A ``Secondary Section'' is a named appendix or a front-matter section of
the Document that deals exclusively with the relationship of the
publishers or authors of the Document to the Document's overall subject
(or to related matters) and contains nothing that could fall directly
within that overall subject.  (Thus, if the Document is in part a
textbook of mathematics, a Secondary Section may not explain any
mathematics.)  The relationship could be a matter of historical
connection with the subject or with related matters, or of legal,
commercial, philosophical, ethical or political position regarding
them.

The ``Invariant Sections'' are certain Secondary Sections whose titles
are designated, as being those of Invariant Sections, in the notice
that says that the Document is released under this License.  If a
section does not fit the above definition of Secondary then it is not
allowed to be designated as Invariant.  The Document may contain zero
Invariant Sections.  If the Document does not identify any Invariant
Sections then there are none.

The ``Cover Texts'' are certain short passages of text that are listed,
as Front-Cover Texts or Back-Cover Texts, in the notice that says that
the Document is released under this License.  A Front-Cover Text may
be at most 5 words, and a Back-Cover Text may be at most 25 words.

A ``Transparent'' copy of the Document means a machine-readable copy,
represented in a format whose specification is available to the
general public, that is suitable for revising the document
straightforwardly with generic text editors or (for images composed of
pixels) generic paint programs or (for drawings) some widely available
drawing editor, and that is suitable for input to text formatters or
for automatic translation to a variety of formats suitable for input
to text formatters.  A copy made in an otherwise Transparent file
format whose markup, or absence of markup, has been arranged to thwart
or discourage subsequent modification by readers is not Transparent.
An image format is not Transparent if used for any substantial amount
of text.  A copy that is not ``Transparent'' is called ``Opaque''.

Examples of suitable formats for Transparent copies include plain
ASCII without markup, Texinfo input format, \LaTeX\ input format, SGML
or XML using a publicly available DTD, and standard-conforming simple
HTML, PostScript or PDF designed for human modification.  Examples of
transparent image formats include PNG, XCF and JPG.  Opaque formats
include proprietary formats that can be read and edited only by
proprietary word processors, SGML or XML for which the DTD and/or
processing tools are not generally available, and the
machine-generated HTML, PostScript or PDF produced by some word
processors for output purposes only.

The ``Title Page'' means, for a printed book, the title page itself,
plus such following pages as are needed to hold, legibly, the material
this License requires to appear in the title page.  For works in
formats which do not have any title page as such, ``Title Page'' means
the text near the most prominent appearance of the work's title,
preceding the beginning of the body of the text.

A section ``Entitled XYZ'' means a named subunit of the Document whose
title either is precisely XYZ or contains XYZ in parentheses following
text that translates XYZ in another language.  (Here XYZ stands for a
specific section name mentioned below, such as ``Acknowledgements'',
``Dedications'', ``Endorsements'', or ``History''.)  To ``Preserve the Title''
of such a section when you modify the Document means that it remains a
section ``Entitled XYZ'' according to this definition.

The Document may include Warranty Disclaimers next to the notice which
states that this License applies to the Document.  These Warranty
Disclaimers are considered to be included by reference in this
License, but only as regards disclaiming warranties: any other
implication that these Warranty Disclaimers may have is void and has
no effect on the meaning of this License.


\subsection{Verbatim Copying}
\label{verbatim}

You may copy and distribute the Document in any medium, either
commercially or noncommercially, provided that this License, the
copyright notices, and the license notice saying this License applies
to the Document are reproduced in all copies, and that you add no other
conditions whatsoever to those of this License.  You may not use
technical measures to obstruct or control the reading or further
copying of the copies you make or distribute.  However, you may accept
compensation in exchange for copies.  If you distribute a large enough
number of copies you must also follow the conditions in
section~\ref{copying}.

You may also lend copies, under the same conditions stated above, and
you may publicly display copies.


\subsection{Copying in Quantity}
\label{copying}

If you publish printed copies (or copies in media that commonly have
printed covers) of the Document, numbering more than 100, and the
Document's license notice requires Cover Texts, you must enclose the
copies in covers that carry, clearly and legibly, all these Cover
Texts: Front-Cover Texts on the front cover, and Back-Cover Texts on
the back cover.  Both covers must also clearly and legibly identify
you as the publisher of these copies.  The front cover must present
the full title with all words of the title equally prominent and
visible.  You may add other material on the covers in addition.
Copying with changes limited to the covers, as long as they preserve
the title of the Document and satisfy these conditions, can be treated
as verbatim copying in other respects.

If the required texts for either cover are too voluminous to fit
legibly, you should put the first ones listed (as many as fit
reasonably) on the actual cover, and continue the rest onto adjacent
pages.

If you publish or distribute Opaque copies of the Document numbering
more than 100, you must either include a machine-readable Transparent
copy along with each Opaque copy, or state in or with each Opaque copy
a computer-network location from which the general network-using
public has access to download using public-standard network protocols
a complete Transparent copy of the Document, free of added material.
If you use the latter option, you must take reasonably prudent steps,
when you begin distribution of Opaque copies in quantity, to ensure
that this Transparent copy will remain thus accessible at the stated
location until at least one year after the last time you distribute an
Opaque copy (directly or through your agents or retailers) of that
edition to the public.

It is requested, but not required, that you contact the authors of the
Document well before redistributing any large number of copies, to give
them a chance to provide you with an updated version of the Document.


\subsection{Modifications}
\label{modifications}

You may copy and distribute a Modified Version of the Document under
the conditions of sections~\ref{verbatim} and \ref{copying} above,
provided that you release
the Modified Version under precisely this License, with the Modified
Version filling the role of the Document, thus licensing distribution
and modification of the Modified Version to whoever possesses a copy
of it.  In addition, you must do these things in the Modified Version:

\begin{enumerate}[A.]
\item Use in the Title Page (and on the covers, if any) a title distinct
   from that of the Document, and from those of previous versions
   (which should, if there were any, be listed in the History section
   of the Document).  You may use the same title as a previous version
   if the original publisher of that version gives permission.
\item List on the Title Page, as authors, one or more persons or entities
   responsible for authorship of the modifications in the Modified
   Version, together with at least five of the principal authors of the
   Document (all of its principal authors, if it has fewer than five),
   unless they release you from this requirement.
\item State on the Title page the name of the publisher of the
   Modified Version, as the publisher.
\item Preserve all the copyright notices of the Document.
\item Add an appropriate copyright notice for your modifications
   adjacent to the other copyright notices.
\item Include, immediately after the copyright notices, a license notice
   giving the public permission to use the Modified Version under the
   terms of this License, in the form shown in the Addendum below.
\item Preserve in that license notice the full lists of Invariant Sections
   and required Cover Texts given in the Document's license notice.
\item Include an unaltered copy of this License.
\item Preserve the section Entitled ``History'', Preserve its Title, and add
   to it an item stating at least the title, year, new authors, and
   publisher of the Modified Version as given on the Title Page.  If
   there is no section Entitled ``History'' in the Document, create one
   stating the title, year, authors, and publisher of the Document as
   given on its Title Page, then add an item describing the Modified
   Version as stated in the previous sentence.
\item Preserve the network location, if any, given in the Document for
   public access to a Transparent copy of the Document, and likewise
   the network locations given in the Document for previous versions
   it was based on.  These may be placed in the ``History'' section.
   You may omit a network location for a work that was published at
   least four years before the Document itself, or if the original
   publisher of the version it refers to gives permission.
\item For any section Entitled ``Acknowledgements'' or ``Dedications'',
   Preserve the Title of the section, and preserve in the section all
   the substance and tone of each of the contributor acknowledgements
   and/or dedications given therein.
\item Preserve all the Invariant Sections of the Document,
   unaltered in their text and in their titles.  Section numbers
   or the equivalent are not considered part of the section titles.
\item Delete any section Entitled ``Endorsements''.  Such a section
   may not be included in the Modified Version.
\item Do not retitle any existing section to be Entitled ``Endorsements''
   or to conflict in title with any Invariant Section.
\item Preserve any Warranty Disclaimers.

\end{enumerate}

If the Modified Version includes new front-matter sections or
appendices that qualify as Secondary Sections and contain no material
copied from the Document, you may at your option designate some or all
of these sections as invariant.  To do this, add their titles to the
list of Invariant Sections in the Modified Version's license notice.
These titles must be distinct from any other section titles.

You may add a section Entitled ``Endorsements'', provided it contains
nothing but endorsements of your Modified Version by various
parties--for example, statements of peer review or that the text has
been approved by an organization as the authoritative definition of a
standard.

You may add a passage of up to five words as a Front-Cover Text, and a
passage of up to 25 words as a Back-Cover Text, to the end of the list
of Cover Texts in the Modified Version.  Only one passage of
Front-Cover Text and one of Back-Cover Text may be added by (or
through arrangements made by) any one entity.  If the Document already
includes a cover text for the same cover, previously added by you or
by arrangement made by the same entity you are acting on behalf of,
you may not add another; but you may replace the old one, on explicit
permission from the previous publisher that added the old one.

The author(s) and publisher(s) of the Document do not by this License
give permission to use their names for publicity for or to assert or
imply endorsement of any Modified Version.


\subsection{Combining Documents}
\label{combining}

You may combine the Document with other documents released under this
License, under the terms defined in section~\ref{modifications}
above for modified
versions, provided that you include in the combination all of the
Invariant Sections of all of the original documents, unmodified, and
list them all as Invariant Sections of your combined work in its
license notice, and that you preserve all their Warranty Disclaimers.

The combined work need only contain one copy of this License, and
multiple identical Invariant Sections may be replaced with a single
copy.  If there are multiple Invariant Sections with the same name but
different contents, make the title of each such section unique by
adding at the end of it, in parentheses, the name of the original
author or publisher of that section if known, or else a unique number.
Make the same adjustment to the section titles in the list of
Invariant Sections in the license notice of the combined work.

In the combination, you must combine any sections Entitled ``History''
in the various original documents, forming one section Entitled
``History''; likewise combine any sections Entitled ``Acknowledgements'',
and any sections Entitled ``Dedications''.  You must delete all sections
Entitled ``Endorsements''.


\subsection{Collections of Documents}
\label{collections}

You may make a collection consisting of the Document and other documents
released under this License, and replace the individual copies of this
License in the various documents with a single copy that is included in
the collection, provided that you follow the rules of this License for
verbatim copying of each of the documents in all other respects.

You may extract a single document from such a collection, and distribute
it individually under this License, provided you insert a copy of this
License into the extracted document, and follow this License in all
other respects regarding verbatim copying of that document.


\subsection{Aggregation with Independent Works}
\label{aggregation}

A compilation of the Document or its derivatives with other separate
and independent documents or works, in or on a volume of a storage or
distribution medium, is called an ``aggregate'' if the copyright
resulting from the compilation is not used to limit the legal rights
of the compilation's users beyond what the individual works permit.
When the Document is included in an aggregate, this License does not
apply to the other works in the aggregate which are not themselves
derivative works of the Document.

If the Cover Text requirement of section~\ref{copying} is applicable to
these copies of the Document, then if the Document is less than one half
of the entire aggregate, the Document's Cover Texts may be placed on
covers that bracket the Document within the aggregate, or the
electronic equivalent of covers if the Document is in electronic form.
Otherwise they must appear on printed covers that bracket the whole
aggregate.


\subsection{Translation}
\label{translation}

Translation is considered a kind of modification, so you may
distribute translations of the Document under the terms of
section~\ref{modifications}.
Replacing Invariant Sections with translations requires special
permission from their copyright holders, but you may include
translations of some or all Invariant Sections in addition to the
original versions of these Invariant Sections.  You may include a
translation of this License, and all the license notices in the
Document, and any Warranty Disclaimers, provided that you also include
the original English version of this License and the original versions
of those notices and disclaimers.  In case of a disagreement between
the translation and the original version of this License or a notice
or disclaimer, the original version will prevail.

If a section in the Document is Entitled ``Acknowledgements'',
``Dedications'', or ``History'', the requirement
(section~\ref{modifications}) to Preserve
its Title (section~\ref{applicability}) will typically require
changing the actual title.


\subsection{Termination}
\label{termination}

You may not copy, modify, sublicense, or distribute the Document except
as expressly provided for under this License.  Any other attempt to
copy, modify, sublicense or distribute the Document is void, and will
automatically terminate your rights under this License.  However,
parties who have received copies, or rights, from you under this
License will not have their licenses terminated so long as such
parties remain in full compliance.

%------------------------------------------------------------------------------

%==============================================================================
%------------------------------------------------------------------------------

%% -*- LaTeX -*-

\ifhtml\else
  \newpage
  \pagenumbering{roman}
  \setcounter{page}{1}
  \printindex
\fi


\end{document}
