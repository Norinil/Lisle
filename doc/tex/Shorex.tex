%% -*- LaTeX -*-

\begin{classpage}{Shorex}

A shorex is a {\em sh}ared {\em or} {\em ex}clusive access
device, and is usefull for protecting shared data structures from
concurrent modifications.
In literature shorex is also called read/write-lock, or rwlock

A shorex has three possible states: unlocked -- not owned by any
thread --, shared locked -- owned by one or more threads --, and
exclusively locked -- owned by exactly one thread. A shorex can be
owned by multiple threads in shared mode, and owned by a single thread
in exclusive mode. A thread attempting to lock an exclusively locked
shorex, or a thread attempting to exclusive lock a locked shorex is
suspended until the shorex returns to the unlocked state.

\mansection{Synopsis}
\begin{mansynopsis}
#include <lisle/Shorex>

namespace lisle {
enum Access { shared, exclusive };
class Shorex
{
public:
  ~Shorex ()
    throw (permission);
  Shorex ()
    throw (resource);
  bool trylock (Access access)
    throw ();
  void lock (Access access)
    throw (deadlock);
  void unlock ()
    throw (permission);
  bool tryacquire (Access access)
    throw ();
  void acquire (Access access)
    throw (deadlock);
  void release ()
    throw (permission);
private:
  // @textnit$Disable cloning?
  Shorex (const Shorex&);
  Shorex& operator = (const Shorex&);
};
}
\end{mansynopsis}

\mansection{Description}
\begin{mandescription}
  \destructor
  Destroys \farg{this} shorex, freeing all resources it might hold. On
  entrance \farg{this} shorex must be unlocked by all owners.
  \begin{exception}
    \item[permission] is thrown if \farg{this}
      shorex was not unlocked by all owners.
  \end{exception}

  \constructor{}
  Constructs \farg{this} shorex.
  \begin{exception}
    \item[resource] is thrown if there are not
      enough system resources to create \farg{this} shorex.
  \end{exception}

  \function{void}{lock,acquire}{Access \farg{access}}
  Applies a lock on \farg{this} shorex.\\
  The given \farg{access} influences the locking as follows:
  \begin{description}
    \item[\code{shared}] applies a shared lock to \farg{this}
      shorex. The calling thread acquires \farg{this} shorex if it is
      not already held exclusively by another thread. Otherwise the
      calling thread blocks until it can acquire the lock.\\
      If the calling thread already holds a lock on \farg{this} shorex
      a deadlock exception is thrown.
    \item[\code{exclusive}] applies an exclusive lock to
      \farg{this} shorex. The calling thread acquires \farg{this}
      shorex if no other thread holds \farg{this} shorex (shared or
      exclusive). Otherwise the calling thread blocks until it can
      acquire the lock.\\
      If the calling thread already holds a lock on \farg{this} shorex
      a deadlock exception is thrown.
  \end{description}
  \begin{exception}
    \item[deadlock] is thrown if a calling thread
      already holding a lock on \farg{this} shorex attempts to relock
      it.
  \end{exception}

  \function{bool}{trylock,tryacquire}{Access \farg{access}}
  These functions behave identically to
  \code{Shorex::lock(\farg{access})}, except that they don't block the
  calling thread if \farg{this} shorex can not be locked . If the
  calling thread could lock \farg{this} shorex \code{true} is
  returned, else these functions return
  \code{false}.\\
  Note that contrary to \code{Shorex::lock()} nothing tells if the
  locking of \farg{this} shorex failed because of locking logic or
  lack of resources.

  \function{void}{unlock,release}{}
  Unlocks \farg{this} shorex. On entrance to this function \farg{this}
  shorex is assumed to be locked by the calling thread.
  \begin{exception}
    \item[permission] is thrown if \farg{this}
      shorex was not locked by the calling thread.
  \end{exception}
\end{mandescription}

\mansection{Cancellation}
\code{Shorex::lock(exclusive)} is a cancellation point.

\mansection{Async-Signal Safety}
The \code{Shorex} member functions are not async-signal safe, and
should not be called from a signal handler. In particular, calling
\code{Shorex::lock()} or \code{Shorex::unlock()} from a signal handler
may deadlock the calling thread.

\end{classpage}
