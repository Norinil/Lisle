%% -*- LaTeX -*-

%%=============================================================================

\begin{manpage}{Creation}
\let\savedmanlayout=\manlayout
\renewcommand{\manlayout}{vcompact}

Function \code{create} creates a new thread of control that executes concurrently with the calling thread.
The thread's main function and arguments are given by means of a \class{Thread} argument, while the thread scheduling policy is given with an optional \class{Schedule} argument.

A total of four different types of thread functions can be created,
for threads with or without arguments, or for threads returning data
or threads which return nothing:
\begin{enumerate}
  \item \code{void \textit{thread} ()}
  \item \code{void \textit{thread} (Handle<\textit{TAT}>)}
  \item \code{Handle<\textit{TRT}> \textit{thread} ()}
  \item \code{Handle<\textit{TRT}> \textit{thread} (Handle<\textit{TAT}>)}
\end{enumerate}
Arguments and/or returns are passed through handles (see section~\manref{sec:Handle}).
Handles are kind of smart pointers with
thread safe memory allocation and deallocation. If more complex
synchronization than memory management is needed in thread arguments
or returns, data stored in the handle will need a synchronization
device, like one documented in
Chapter~\ref{cha:Synchronization-Devices}.

\mansection{Synopsis}
\begin{mansynopsis}
#include <lisle/create>

Thrid create (const Thread& start,
              const Schedule& schedule = Schedule())
  throw (resource, permission, virthread);
\end{mansynopsis}

\mansection{Description}
\begin{mandescription}
  \function{Thrid}{create}{%
    const Thread\& \farg{start},%
    const Schedule\& \farg{schedule}}
  Creates a new thread of control that executes concurrently with the
  calling thread. Returns a \code{Thrid} which identifies the
  created thread. Don't loose this \code{Thrid} if you intend to
  join the created thread (what you have to if the created thread is
  not in detached state). Argument \farg{start} describes the thread's
  main function and arguments, while the optional argument
  \farg{schedule} describes the scheduling policy
  in which the created thread should run.
  \begin{exception}
    \item[resource] is thrown if there are
      not enough system resources to create the new thread. This
      happens if either too many threads are running in the same
      process, or if there is a shortage in memory resources.
    \item[virthread] is thrown if
      \farg{start}\code{.main()} returns \code{NULL}. In other words
      if a default \farg{start} argument was given, or if a
      \code{NULL} function pointer was given to construct
      \farg{start}.
    \item[permission] is thrown if the
      process owner isn't allowed to run a thread with a given
      scheduling policy given in the \farg{schedule} argument. Only
      the super user can run threads whith a scheduling policy other
      than \code{Schedule::Regular}.
  \end{exception}
\end{mandescription}

\mansection{Example}
The following example shows the four different types of thread
functions that can be created, how to start them, and if needed how to
give them arguments, and if they return a result how to retrieve the
handle to the return.

\VerbatimInput[frame=lines,labelposition=topline,label=\textnit{Thread
  creation},framesep=3ex,xleftmargin=2.5em,numbers=left,tabsize=2]
  {../example/create-thread.cpp}

\renewcommand{\manlayout}{\savedmanlayout}
\end{manpage}

%%=============================================================================

% \newpage
% \index{Thread::exit|(}

% \section{Thread::exit}
% \label{sec:Thread::exit}

% \nsThreadNote{exit}{function}

% \begin{mansec}{Synopsis}
% \begin{Synopsis}
% #include <lyric/Thread>

% void Thread::exit (void* retdataptr = NULL);
% \end{Synopsis}
% \end{mansec}

% \begin{mansec}{Description}
%   \begin{manfunc}{void Thread::exit (void* \farg{retdataptr})}
%   \end{manfunc}
% \end{mansec}

% \index{Thread::exit|)}

% %%=============================================================================

% \newpage
% \index{Thread::set {\it (cancel behaviour)}|(}

% \section{Thread::set {\it (cancel behaviour)}}
% \label{sec:Thread::set(Thread::Cancel)}

% \code{Thread::set(\farg{cancel})} sets the calling thread's cancel
% behaviour. See~\lhmref{sec:Thread::Cancel} for the
% \code{Thread::Cancel} documentation.
% behaviour.

% \nsThreadNote{set({\it cancel behaviour})}{function}

% \begin{mansec}{Synopsis}
% \begin{Synopsis}
% #include <lyric/Thread>

% void Thread::set (const Thread::Cancel& cancel);
% \end{Synopsis}
% \end{mansec}

% \begin{mansec}{Description}
%   \begin{manfunc}{void Thread::set (const Thread::Cancel\& \farg{cancel})}
%     Set the calling thread's cancel behaviour to the given
%     \farg{cancel} argument.
%   \end{manfunc}
% \end{mansec}

% \index{Thread::set {\it (cancel behaviour)}|)}

% %%=============================================================================

% \vspace*{10ex}
% \index{Thread::get {\it (cancel behaviour)}|(}

% \section{Thread::get {\it (cancel behaviour)}}
% \label{sec:Thread::get(Thread::Cancel)}

% \code{Thread::get(\farg{cancel})} grabs the calling thread's cancel
% behaviour. See~\lhmref{sec:Thread::Cancel} for the
% \code{Thread::Cancel} documentation.

% \nsThreadNote{get({\it cancel behaviour})}{function}

% \begin{mansec}{Synopsis}
% \begin{Synopsis}
% #include <lyric/Thread>

% void Thread::get (Thread::Cancel& cancel);
% \end{Synopsis}
% \end{mansec}

% \begin{mansec}{Description}
%   \begin{manfunc}{void Thread::get (Thread::Cancel\& \farg{cancel})}
%     Grabs the calling thread's cancel behaviour and returns it in
%     \farg{cancel}.
%   \end{manfunc}
% \end{mansec}

% \index{Thread::get {\it (cancel behaviour)}|)}

% %%=============================================================================

% \newpage
% \index{Thread::testcancel|(}

% \section{Thread::testcancel}
% \label{sec:Thread::testcancel}

% \code{Thread::cancel} does nothing except testing for pending
% cancellation and executing it. Its purpose is to introduce explicit
% checks for cancellation in long sequences of code that do not call
% cancellation point functions otherwise.

% \nsThreadNote{testcancel}{function}

% \begin{mansec}{Synopsis}
% \begin{Synopsis}
% #include <lyric/Thread>

% void Thread::testcancel ();
% \end{Synopsis}
% \end{mansec}

% \begin{mansec}{Description}
%   \begin{manfunc}{Thread::testcancel ()}
%     Tests for pending cancellation and executes it.
%   \end{manfunc}
% \end{mansec}

% \index{Thread::testcancel|)}

% %%=============================================================================

% \newpage
% \index{Thread::yield|(}

% \section{Thread::yield}
% \label{sec:Thread::yield}

% A thread can relinquish the processor voluntarily without blocking by
% calling \code{Thread::yield()}. The thread will then be moved to the
% end of the queue for its static priority and a new thread gets to run.

% \nsThreadNote{yield}{function}

% \begin{mansec}{Synopsis}
% \begin{Synopsis}
% #include <lyric/Thread>

% void Thread::yield ();
% \end{Synopsis}
% \end{mansec}

% \begin{mansec}{Description}
%   \begin{manfunc}{void Thread::yield ()}
%     The calling thread relinquishes the processor, giving another
%     thread a higher chance to run.\\
%     If the current thread is the only thread in the highest priority
%     list at that time, this thread will continue to run after a call
%     to \code{Thread::yield()}.
%   \end{manfunc}
% \end{mansec}

% \index{Thread::yield|)}

%%=============================================================================
