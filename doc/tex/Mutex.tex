%% -*- LaTeX -*-

\begin{classpage}{Mutex}
  
A mutex is a {\em mut}ual {\em ex}clusion device, and is useful for
protecting shared data structures from concurrent modifications, and
implementing critical sections and monitors.

A mutex has two possible states: unlocked -- not owned by any thread
--, and locked -- owned by one thread. A mutex can never be owned by
two different threads simultaneously.  A thread attempting to lock a
mutex that is already locked by another thread is suspended until the
owning thread unlocks the mutex first.

The mutex has error checking enabled. If a thread attempts to relock
a mutex that it has already locked, a deadlock exception is thrown.
If a thread attempts to unlock a mutex that it has not locked or a mutex which is unlocked, a permission exception is thrown.

\mansection{Synopsis}
\begin{mansynopsis}
#include <lisle/Mutex>

class Mutex
{
public:
  ~Mutex ()
    throw (permission);
  Mutex ()
    throw (resource);
  bool trylock ()
    throw ();
  void lock ()
    throw (deadlock);
  void unlock ()
    throw (permission);
  bool tryacquire ()
    throw ();
  void acquire ()
    throw (deadlock);
  void release ()
    throw (permission);
private:
  // @textnit$Disable cloning?
  Mutex (const Mutex&);
  Mutex& operator = (const Mutex&);
};
\end{mansynopsis}

\mansection{Description}
\begin{mandescription}
  \destructor
  Destroys \farg{this} mutex, freeing all resources it might
  hold. On entrance \farg{this} mutex must be unlocked.
  \begin{exception}
    \item[permission] is thrown if \farg{this} mutex was
      not unlocked before being destroyed.
  \end{exception}

  \constructor{}
  Constructs \farg{this} mutex.
  \begin{exception}
    \item[resource] is thrown if there are not
      enough system resources to create a new mutex.
  \end{exception}

  \function{void}{lock,acquire}{}
  Locks \farg{this} mutex.\\
  If \farg{this} mutex is currently unlocked, it becomes locked and
  owned by the calling thread, and \code{lock} returns
  immediately.\\
  If \farg{this} mutex is already locked by another thread,
  \code{lock} suspends the calling thread until \farg{this} mutex is
  unlocked.\\
  If \farg{this} mutex is already locked by the calling thread, the
  function throws a \code{deadlock} exception and returns immediately.
  \begin{exception}
    \item[deadlock] is thrown if a calling thread tries to lock
    \farg{this} mutex when it is already owned by the thread.    
  \end{exception}

  \function{bool}{trylock,tryacquire}{}
  These functions behave identically to \code{lock()}, except
  that they don't block the calling thread if \farg{this} mutex is
  already locked by a thread (including the calling thread). If the
  calling thread could lock \farg{this} mutex \code{true} is returned,
  else these functions return \code{false}.

  \function{void}{unlock,release}{}
  Unlocks \farg{this} mutex. On entrance to this function \farg{this}
  mutex is assumed to be locked and owned by the calling
  thread.\\
  If \farg{this} mutex is not owned by the calling thread an
  \code{permission} exception is thrown.
  \begin{exception}
    \item[permission] is thrown if the calling thread does not own
    \farg{this} mutex.
  \end{exception}
\end{mandescription}

\mansection{Cancellation}
  None of the mutex functions is a cancellation point, not even
  \code{lock()}, in spite of the fact that it can suspend a
  thread for arbitrary durations. This way, the status of mutexes at
  cancellation points is predictable.

\mansection{Async-Signal Safety}
  The mutex functions are not async-signal safe. What this means is
  that they should not be called from a signal handler. In particular,
  calling \code{lock()} or \code{unlock()} from a signal
  handler may deadlock the calling thread.

\end{classpage}
