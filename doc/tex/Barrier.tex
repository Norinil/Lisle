%% -*- LaTeX -*-

\begin{classpage}{Barrier}

Barriers are multiple threads synchronisation devices. A barrier
blocks waiting threads until the given number of clients have reached
it. Barrier objects are reusable. This means that the same barrier
object can be waited for sequentially multiple times by a given
thread.

\mansection{Synopsis}
\begin{mansynopsis}
#include <lisle/Barrier>

class Barrier
{
public:
  ~Barrier ()
    throw (permission);
  Barrier (size_t clients)
    throw (resource);
  void wait ()
    throw ();
private:
  // @textnit$Disable cloning?
  Barrier (const Barrier& barrier);
  Barrier& operator = (const Barrier& barrier);
};
\end{mansynopsis}

\mansection{Description}
\begin{mandescription}
  \destructor
  Destroys \farg{this} barrier, releasing all resources it might
  hold. No threads must be waiting on \farg{this} barrier on
  entrance.
  \begin{exception}
    \item[permission] is thrown if there are
      waiting threads for \farg{this} barrier.
  \end{exception}

  \constructor{size\_t \farg{clients}}
  Constructs \farg{this} barrier with the given number of
  \farg{clients}. The created barrier will block waiting threads
  until the given number of \farg{clients} is reached.
  \begin{exception}
    \item[resource] is thrown if there are not
      enough system resources to create \farg{this} new barrier.
  \end{exception}

  \function{void}{wait}{}
  Suspends the calling thread until the expected number of \farg{client} threads are waiting on \farg{this} barrier. 
  Once the expected number of \farg{client}s threads have reached \farg{this} barrier all waiting threads are restarted.
\end{mandescription}

\mansection{Cancellation}
None of the barrier functions is a cancellation point.

\mansection{Async-Signal Safety}
The \code{Barrier} member functions are not async-signal safe, and
should not be called from a signal handler. Calling
\code{Barrier::wait()} from a signal handler may deadlock the
calling thread.

\end{classpage}
