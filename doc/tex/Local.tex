%% -*- LaTeX -*-

\begin{classpage}{Local}

Programs sometimes need global or static variables that have different
values in different threads. Since threads share one memory space,
this cannot be achieved with regular variables. The
\class{Local} class is the \lisle threads answer to this need.

Each thread possesses a private memory block, the thread-specific data
area, or TSD area for short. This area is accessible with objects of
the \class{Local} class. Each instance of the
\class{Local} class is common to all threads, but the value
associated with a given \code{Local} can be different in each
thread.

\mansection{Synopsis}
\begin{mansynopsis}
#include <lisle/Local>

template <typename T>
class Local
{
public:
  ~Local ();
  Local ()
    throw (resource);
  T& operator = (const T& data)
    throw (resource);
  operator T* ()
    throw ();
  operator T* () const
    throw ();
  T* operator -> ()
    throw ();
  T* operator -> () const
    throw ();
private:
  // @textnit$Disable cloning?
  Local (const Local&);
  Local& operator = (const Local&);
};
\end{mansynopsis}

\mansection{Description}
\begin{mandescription}
  \destructor
  Destroys \farg{this} TSD, releasing all used resources.

  \constructor{}
  Constructs \farg{this} TSD. Notice the abscence of a constructor
  with default argument. It is not possible to construct a TSD with an
  initial value other than \code{NULL}. Each thread \emph{must}
  initialize its ``instance'' of \farg{this} TSD with the assignment
  operator (see below).
  \begin{exception}
    \item[resource] is thrown if the TSD area space is
      exhausted.
  \end{exception}

  \operator{T\&}{=}{=}{const T\& data}
  Assigns some \farg{data} to \farg{this} TSD. Returns a reference to
  the stored data for assignment operation chaining.
  
  This operator uses \code{T::operator =} to store \farg{data} in
  \farg{this} TSD. Depending on the implementation of the \code{=}
  operator for a given \code{T} class, especially for containers,
  what will finally be stored in \farg{this} TSD may vary (data or
  pointer).
  
  Assigning data to \farg{this} TSD is \emph{mandatory} before using
  data stored in \farg{this} TSD. At least one assignment has to be
  done before \farg{this} TSD can be used with the casting operators
  described below.  As analogy consider \farg{this} TSD as a pointer,
  set to \code{NULL}. As long as a pointer is not initialized with
  \code{new}, \code{malloc} or assignment to an initialized pointer it
  can't be used without the risk of memory leaks. It is the same for
  TSDs. The only difference is that only this assigment operator is
  available for initializing \farg{this} TSD. Note that the
  initialization (aka assignment) \emph{must} be done once in
  \emph{each} thread.

  \begin{exception}
    \item[resource] is thrown if there was not enough
      memory to store one item of type \code{T} in the calling
      thread's memory space (the shared memory space, not the TSD
      area).
  \end{exception}

  \operator{}{T*}{T*}{}
  Returns the pointer to the data stored in \farg{this} TSD.
  \operator[const]{}{T*}{const T*}{}
  Returns the pointer to the data stored in \farg{this} TSD.

  \operator{T*}{->}{-$>$}{}
  Returns the pointer to the data stored in \farg{this} TSD, giving
  access to data and function members if \code{T} is a structure or a
  class.
  \operator[const]{T*}{->}{-$>$}{}
  Returns the pointer to the data stored in \farg{this} TSD, giving
  access to data and function members if \code{T} is a structure or a
  class.

\end{mandescription}

\end{classpage}
